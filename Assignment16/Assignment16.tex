\documentclass[journal,12pt]{IEEEtran}
\usepackage{longtable}
\usepackage{setspace}
\usepackage{gensymb}
\singlespacing
\usepackage[cmex10]{amsmath}
\newcommand\myemptypage{
	\null
	\thispagestyle{empty}
	\addtocounter{page}{-1}
	\newpage
}
\usepackage{amsthm}
\usepackage{mdframed}
\usepackage{mathrsfs}
\usepackage{txfonts}
\usepackage{stfloats}
\usepackage{bm}
\usepackage{cite}
\usepackage{cases}
\usepackage{subfig}

\usepackage{longtable}
\usepackage{multirow}

\usepackage{enumitem}
\usepackage{mathtools}
\usepackage{steinmetz}
\usepackage{tikz}
\usepackage{circuitikz}
\usepackage{verbatim}
\usepackage{tfrupee}
\usepackage[breaklinks=true]{hyperref}
\usepackage{graphicx}
\usepackage{tkz-euclide}

\usetikzlibrary{calc,math}
\usepackage{listings}
    \usepackage{color}                                            %%
    \usepackage{array}                                            %%
    \usepackage{longtable}                                        %%
    \usepackage{calc}                                             %%
    \usepackage{multirow}                                         %%
    \usepackage{hhline}                                           %%
    \usepackage{ifthen}                                           %%
    \usepackage{lscape}     
\usepackage{multicol}
\usepackage{chngcntr}

\DeclareMathOperator*{\Res}{Res}

\renewcommand\thesection{\arabic{section}}
\renewcommand\thesubsection{\thesection.\arabic{subsection}}
\renewcommand\thesubsubsection{\thesubsection.\arabic{subsubsection}}

\renewcommand\thesectiondis{\arabic{section}}
\renewcommand\thesubsectiondis{\thesectiondis.\arabic{subsection}}
\renewcommand\thesubsubsectiondis{\thesubsectiondis.\arabic{subsubsection}}


\hyphenation{op-tical net-works semi-conduc-tor}
\def\inputGnumericTable{}                                 %%

\lstset{
%language=C,
frame=single, 
breaklines=true,
columns=fullflexible
}
\begin{document}
\onecolumn

\newtheorem{theorem}{Theorem}[section]
\newtheorem{problem}{Problem}
\newtheorem{proposition}{Proposition}[section]
\newtheorem{lemma}{Lemma}[section]
\newtheorem{corollary}[theorem]{Corollary}
\newtheorem{example}{Example}[section]
\newtheorem{definition}[problem]{Definition}

\newcommand{\BEQA}{\begin{eqnarray}}
\newcommand{\EEQA}{\end{eqnarray}}
\newcommand{\define}{\stackrel{\triangle}{=}}
\bibliographystyle{IEEEtran}
\raggedbottom
\setlength{\parindent}{0pt}
\providecommand{\mbf}{\mathbf}
\providecommand{\pr}[1]{\ensuremath{\Pr\left(#1\right)}}
\providecommand{\qfunc}[1]{\ensuremath{Q\left(#1\right)}}
\providecommand{\sbrak}[1]{\ensuremath{{}\left[#1\right]}}
\providecommand{\lsbrak}[1]{\ensuremath{{}\left[#1\right.}}
\providecommand{\rsbrak}[1]{\ensuremath{{}\left.#1\right]}}
\providecommand{\brak}[1]{\ensuremath{\left(#1\right)}}
\providecommand{\lbrak}[1]{\ensuremath{\left(#1\right.}}
\providecommand{\rbrak}[1]{\ensuremath{\left.#1\right)}}
\providecommand{\cbrak}[1]{\ensuremath{\left\{#1\right\}}}
\providecommand{\lcbrak}[1]{\ensuremath{\left\{#1\right.}}
\providecommand{\rcbrak}[1]{\ensuremath{\left.#1\right\}}}
\theoremstyle{remark}
\newtheorem{rem}{Remark}
\newcommand{\sgn}{\mathop{\mathrm{sgn}}}
\providecommand{\abs}[1]{\left\vert#1\right\vert}
\providecommand{\res}[1]{\Res\displaylimits_{#1}} 
\providecommand{\norm}[1]{\left\lVert#1\right\rVert}
%\providecommand{\norm}[1]{\lVert#1\rVert}
\providecommand{\mtx}[1]{\mathbf{#1}}
\providecommand{\mean}[1]{E\left[ #1 \right]}
\providecommand{\fourier}{\overset{\mathcal{F}}{ \rightleftharpoons}}
%\providecommand{\hilbert}{\overset{\mathcal{H}}{ \rightleftharpoons}}
\providecommand{\system}{\overset{\mathcal{H}}{ \longleftrightarrow}}
	%\newcommand{\solution}[2]{\textbf{Solution:}{#1}}
\newcommand{\solution}{\noindent \textbf{Solution: }}
\newcommand{\cosec}{\,\text{cosec}\,}
\providecommand{\dec}[2]{\ensuremath{\overset{#1}{\underset{#2}{\gtrless}}}}
\newcommand{\myvec}[1]{\ensuremath{\begin{pmatrix}#1\end{pmatrix}}}
\newcommand{\mydet}[1]{\ensuremath{\begin{vmatrix}#1\end{vmatrix}}}
\numberwithin{equation}{subsection}
\makeatletter
\@addtoreset{figure}{problem}
\makeatother
\let\StandardTheFigure\thefigure
\let\vec\mathbf
\renewcommand{\thefigure}{\theproblem}
\def\putbox#1#2#3{\makebox[0in][l]{\makebox[#1][l]{}\raisebox{\baselineskip}[0in][0in]{\raisebox{#2}[0in][0in]{#3}}}}
     \def\rightbox#1{\makebox[0in][r]{#1}}
     \def\centbox#1{\makebox[0in]{#1}}
     \def\topbox#1{\raisebox{-\baselineskip}[0in][0in]{#1}}
     \def\midbox#1{\raisebox{-0.5\baselineskip}[0in][0in]{#1}}
\vspace{3cm}
\title{Assignment 16}
\author{Neha Rani\\EE20MTECH14014}
\maketitle
\bigskip
\renewcommand{\thefigure}{\theenumi}
\renewcommand{\thetable}{\theenumi}
%
Download the latex-tikz codes from 
%
\begin{lstlisting}
https://github.com/neharani289/MatrixTheory/Assignment16
\end{lstlisting}
\section{\textbf{Problem}}
%
(hoffman/page226/9) : \\
Give an example of two $4\times 4$ nilpotent matrices which have the same minimal polynomial(they necessarily have the same characteristic polynomial) but which are not similar. 
\bigskip
\section{\textbf{Definitions}}
\renewcommand{\thetable}{1}
\begin{table}[ht!]
\centering
\begin{tabular}{|p{5cm}|p{13cm}|}
    \hline
	\multirow{3}{*}{Characteristic Polynomial} 
	& For an $n\times n$ matrix $\vec{A}$, characteristic polynomial is defined by,\\
	&\\
	& $\qquad\qquad\qquad p\brak{x}=\mydet{x\Vec{I}-\Vec{A}}$\\
	&\\
	\hline
	\multirow{3}{*}{Minimal Polynomial} 
	&\\
	& Minimal polynomial $m\brak{x}$ is the smallest factor of characteristic polynomial\\
	& $p\brak{x}$ such that,\\
	&\\
	& $\qquad \qquad \qquad m\brak{\vec{A}}=0$\\
    \hline
    \multirow{3}{*}{Cayley-Hamilton Theorem}
    &\\
    & If $p\brak{x}$ is the characteristic polynomial of an $n\times n$ matrix $\vec{A}$, then,\\
    &\\
    &$\qquad \qquad \qquad p\brak{\vec{A}}=\vec{0}$\\
    &\\
    \hline
    \multirow{3}{*}{Similar matrices}&\\
    &  Two   matrices $\vec{A}$ and $\vec{B}$ are said to be similar if\\
    &{\begin{enumerate}
        \item $det(\vec{A}) = det(\vec{B})$
        
        \item $tr (\vec{A}) = tr (\vec{B})$
       
        \item $rank(\vec{A})=rank(\vec{B})$
    \end{enumerate}}\\
    \hline
    \multirow{3}{*}{Nilpotent Matrix}&\\
    & A square matrix $\vec{A}$ is called a Nilpotent matrix if there exist a positive integer $'m'$ such that $\vec{A}^{m}=\vec{0}$ and $'m'$ is called Index of nilpotent matrix $\vec{A}$.\\
    &\\
   & The determinant and trace of a nilpotent matrix are always zero.\\
   &\\
    \hline
    
\end{tabular}
\label{table:1}
    \caption{Definitions}
\end{table}
\newpage
\section{\textbf{Solution}}
\renewcommand{\thetable}{2}
\begin{longtable}{|p{5cm}|p{13cm}|}
    \hline
    \multirow{3}{*}{Given} 
    &\\
    & Let $\vec{A}$ and $\vec{B}$ be two nilpotent matrix.\\
    &\\
    &$\qquad\qquad\qquad \vec{A}=\myvec{0&1&0&0\\0&0&0&0\\0&0&0&1\\0&0&0&0}$\\
    &\\
    &$\qquad\qquad\qquad\vec{B}=\myvec{0&0&0&0\\0&0&0&0\\0&0&0&1\\0&0&0&0}$\\
    &\\
    \hline
	\multirow{3}{*}{Characteristic polynomial} 
	& \\
	& For Matrix $\vec{A}$;\\
	& $\qquad\qquad\qquad\qquad p_A\brak{x}=\mydet{x\Vec{I}-\Vec{A}}$\\
	&\\
	& $\qquad\qquad\qquad p_A\brak{x}=\mydet{x&-1&0&0\\0&x&0&0\\0&0&x&-1\\0&0&0&x} = x^4$\\
	&\\
	& For Matrix $\vec{B}$; \\
	& $\qquad\qquad\qquad\qquad p_B\brak{x}=\mydet{x\Vec{I}-\Vec{B}}$\\
	&\\
	& $\qquad\qquad\qquad p_B\brak{x}= \mydet{x&0&0&0\\0&x&0&0\\0&0&x&-1\\0&0&0&x} = x^4$\\
	&\\
	& Therefore, Characteristics polynomial are same for both matrix $\vec{A}$ and $\vec{B}$.\\
	&\\
	\hline
	\multirow{3}{*}{Minimal Polynomial} & \\
	& For Matrix $\vec{A}$;\\
	&\\
    & $p_A(x)$ = $x^4$ \\
    &\\
    & Let, minimal polynomial of $\vec{A}$ is $m_A(x)$\\
    &\\
    & $m_A(x)$ always divide $p_A(x)$ \\
    &\\
    & $m_A(x)$ = $\{x,x^2,x^3,x^4\}$\\
    &\\
    &  Minimal polynomial always annihilates its matrix.\\
    &\\
    \hline
    & $\qquad\qquad\qquad m_A(\vec{A}) = \vec{A} =\myvec{0&1&0&0\\0&0&0&0\\0&0&0&1\\0&0&0&0} \neq\vec{0}$\\
    &\\
    &$\qquad\qquad\qquad m_A\brak{\vec{A}}=\vec{A}^2 =\myvec{0&1&0&0\\0&0&0&0\\0&0&0&1\\0&0&0&0}\myvec{0&1&0&0\\0&0&0&0\\0&0&0&1\\0&0&0&0}=\vec{0}$\\
    &\\
    & $ \qquad\qquad\qquad\implies m_A\brak{x} = x^2$\\
    &\\
    & $\implies x^2$ is a minimal polynomial of Matrix $\vec{A}$\\
    &\\
    & For Matrix $\vec{B}$;\\
    &$\qquad\qquad\qquad m_B(\vec{B}) = \vec{B} =\myvec{0&0&0&0\\0&0&0&0\\0&0&0&1\\0&0&0&0} \neq\vec{0}$\\
    &\\
    &$\qquad\qquad\qquad m_B\brak{\vec{B}}=\vec{B}^2 =\myvec{0&0&0&0\\0&0&0&0\\0&0&0&1\\0&0&0&0}\myvec{0&0&0&0\\0&0&0&0\\0&0&0&1\\0&0&0&0}=\vec{0}$\\
    &\\
    & $\qquad\qquad\qquad\implies m_B\brak{x} = x^2$\\
    &\\
    & $\implies x^2$ is a minimal polynomial of Matrix $\vec{B}$\\
    &\\
    & Therefore, minimial polynomial for both Matrix $\vec{A}$ and $\vec{B}$ are same.\\
    &\\
    \hline
	\multirow{3}{*}{}&\\
	Checking whether Matrix $\vec{A}$ & \\
    and $\vec{B}$ are similar .
    & $ \vec{A}=\myvec{0&1&0&0\\0&0&0&0\\0&0&0&1\\0&0&0&0}$\\
    &\\
   & rank of matrix = no of linearly independent row or column vectors in the matrix \\
   &\\
   &  $rank\brak{ \vec{A} }= 2$\\
   &\\
   & $tr(\vec{A})= 0 $\\
   &\\
   & $det(\vec{A})= 0$\\
   &\\
   \hline
   & $ \vec{B}=\myvec{0&0&0&0\\0&0&0&0\\0&0&0&1\\0&0&0&0}$\\
   &\\
   &  $rank\brak{ \vec{B} }= 1$\\
   &\\
   & $tr(\vec{B})= 0 $\\
   &\\
   & $det(\vec{B})= 0$\\
   &\\
   &$\implies rank\brak{\vec{A}} \neq rank\brak{\vec{B}}$\\
   &\\
   & Therefore, Matrix $\vec{A}$ and $\vec{B}$ are not similar.\\
   &\\
	\hline
	\multirow{3}{*}{Conclusion}
	&{\begin{enumerate}
	    \item Characteristics polynomial for both Matrix $\vec{A}$ and $\vec{B}$ are same.
	    
	    \item Minimal polynomial for both matrix are same i.e $m_A\brak{x}=m_B\brak{x}=x^2$.
	    
	    \item Matrix $\vec{A}$ and $\vec{B}$ are not similar .
	\end{enumerate}}\\
	\hline
	\caption{Solution Summary}
    \label{table:2}
\end{longtable}
\end{document}
%
