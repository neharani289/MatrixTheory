\documentclass[journal,12pt]{IEEEtran}
\usepackage{longtable}
\usepackage{setspace}
\usepackage{gensymb}
\singlespacing
\usepackage[cmex10]{amsmath}
\newcommand\myemptypage{
	\null
	\thispagestyle{empty}
	\addtocounter{page}{-1}
	\newpage
}
\usepackage{amsthm}
\usepackage{mdframed}
\usepackage{mathrsfs}
\usepackage{txfonts}
\usepackage{stfloats}
\usepackage{bm}
\usepackage{cite}
\usepackage{cases}
\usepackage{subfig}

\usepackage{longtable}
\usepackage{multirow}

\usepackage{enumitem}
\usepackage{mathtools}
\usepackage{steinmetz}
\usepackage{tikz}
\usepackage{circuitikz}
\usepackage{verbatim}
\usepackage{tfrupee}
\usepackage[breaklinks=true]{hyperref}
\usepackage{graphicx}
\usepackage{tkz-euclide}

\usetikzlibrary{calc,math}
\usepackage{listings}
    \usepackage{color}                                            %%
    \usepackage{array}                                            %%
    \usepackage{longtable}                                        %%
    \usepackage{calc}                                             %%
    \usepackage{multirow}                                         %%
    \usepackage{hhline}                                           %%
    \usepackage{ifthen}                                           %%
    \usepackage{lscape}     
\usepackage{multicol}
\usepackage{chngcntr}

\DeclareMathOperator*{\Res}{Res}

\renewcommand\thesection{\arabic{section}}
\renewcommand\thesubsection{\thesection.\arabic{subsection}}
\renewcommand\thesubsubsection{\thesubsection.\arabic{subsubsection}}

\renewcommand\thesectiondis{\arabic{section}}
\renewcommand\thesubsectiondis{\thesectiondis.\arabic{subsection}}
\renewcommand\thesubsubsectiondis{\thesubsectiondis.\arabic{subsubsection}}


\hyphenation{op-tical net-works semi-conduc-tor}
\def\inputGnumericTable{}                                 %%

\lstset{
%language=C,
frame=single, 
breaklines=true,
columns=fullflexible
}
\begin{document}
\onecolumn

\newtheorem{theorem}{Theorem}[section]
\newtheorem{problem}{Problem}
\newtheorem{proposition}{Proposition}[section]
\newtheorem{lemma}{Lemma}[section]
\newtheorem{corollary}[theorem]{Corollary}
\newtheorem{example}{Example}[section]
\newtheorem{definition}[problem]{Definition}

\newcommand{\BEQA}{\begin{eqnarray}}
\newcommand{\EEQA}{\end{eqnarray}}
\newcommand{\define}{\stackrel{\triangle}{=}}
\bibliographystyle{IEEEtran}
\raggedbottom
\setlength{\parindent}{0pt}
\providecommand{\mbf}{\mathbf}
\providecommand{\pr}[1]{\ensuremath{\Pr\left(#1\right)}}
\providecommand{\qfunc}[1]{\ensuremath{Q\left(#1\right)}}
\providecommand{\sbrak}[1]{\ensuremath{{}\left[#1\right]}}
\providecommand{\lsbrak}[1]{\ensuremath{{}\left[#1\right.}}
\providecommand{\rsbrak}[1]{\ensuremath{{}\left.#1\right]}}
\providecommand{\brak}[1]{\ensuremath{\left(#1\right)}}
\providecommand{\lbrak}[1]{\ensuremath{\left(#1\right.}}
\providecommand{\rbrak}[1]{\ensuremath{\left.#1\right)}}
\providecommand{\cbrak}[1]{\ensuremath{\left\{#1\right\}}}
\providecommand{\lcbrak}[1]{\ensuremath{\left\{#1\right.}}
\providecommand{\rcbrak}[1]{\ensuremath{\left.#1\right\}}}
\theoremstyle{remark}
\newtheorem{rem}{Remark}
\newcommand{\sgn}{\mathop{\mathrm{sgn}}}
\providecommand{\abs}[1]{\left\vert#1\right\vert}
\providecommand{\res}[1]{\Res\displaylimits_{#1}} 
\providecommand{\norm}[1]{\left\lVert#1\right\rVert}
%\providecommand{\norm}[1]{\lVert#1\rVert}
\providecommand{\mtx}[1]{\mathbf{#1}}
\providecommand{\mean}[1]{E\left[ #1 \right]}
\providecommand{\fourier}{\overset{\mathcal{F}}{ \rightleftharpoons}}
%\providecommand{\hilbert}{\overset{\mathcal{H}}{ \rightleftharpoons}}
\providecommand{\system}{\overset{\mathcal{H}}{ \longleftrightarrow}}
	%\newcommand{\solution}[2]{\textbf{Solution:}{#1}}
\newcommand{\solution}{\noindent \textbf{Solution: }}
\newcommand{\cosec}{\,\text{cosec}\,}
\providecommand{\dec}[2]{\ensuremath{\overset{#1}{\underset{#2}{\gtrless}}}}
\newcommand{\myvec}[1]{\ensuremath{\begin{pmatrix}#1\end{pmatrix}}}
\newcommand{\mydet}[1]{\ensuremath{\begin{vmatrix}#1\end{vmatrix}}}
\numberwithin{equation}{subsection}
\makeatletter
\@addtoreset{figure}{problem}
\makeatother
\let\StandardTheFigure\thefigure
\let\vec\mathbf
\renewcommand{\thefigure}{\theproblem}
\def\putbox#1#2#3{\makebox[0in][l]{\makebox[#1][l]{}\raisebox{\baselineskip}[0in][0in]{\raisebox{#2}[0in][0in]{#3}}}}
     \def\rightbox#1{\makebox[0in][r]{#1}}
     \def\centbox#1{\makebox[0in]{#1}}
     \def\topbox#1{\raisebox{-\baselineskip}[0in][0in]{#1}}
     \def\midbox#1{\raisebox{-0.5\baselineskip}[0in][0in]{#1}}
\vspace{3cm}
\title{Assignment 17}
\author{Neha Rani\\EE20MTECH14014}
\maketitle
\bigskip
\renewcommand{\thefigure}{\theenumi}
\renewcommand{\thetable}{\theenumi}
%
Download the latex-tikz codes from 
%
\begin{lstlisting}
https://github.com/neharani289/MatrixTheory/Assignment17
\end{lstlisting}
\section{\textbf{Problem}}
%
(ugcjune/2018/28) : \\

If $\vec{A}$ is a $2\times2$ matrix over $\mathbb{R}$ with $\det\brak{\vec{A}+\vec{I}} = 1+\det\brak{\vec{A}}$, then we can conclude that\\
\begin{enumerate}
    \item $\det\brak{\vec{A}}=0$\\
    \item $\vec{A}=0$\\
    \item  $tr(\vec{A})=0$\\
    \item $\vec{A}$ is non singular.
\end{enumerate}
\section{\textbf{Solution}}
\renewcommand{\thetable}{1}
\begin{longtable}{|p{5cm}|p{13cm}|}
\hline
    \multirow{3}{*}{Given} 
    &\\
     & $\vec{A}$ be a $2\times2$ matrix over $\mathbb{R}$ with\\
     &\\
      &$\qquad\qquad \qquad \det\brak{\vec{A}+\vec{I}}=1+\det(\vec{A})$\\
      &\\
     \hline
     \multirow{3}{*}{Explanation} &\\
     & If $\vec{X}$ is an eigen vector of matrix $\vec{A}$
     corresponding to the eigen value $\lambda$ i.e \\
     &\\
     & $\qquad\qquad\qquad\vec{A}\vec{X}=\lambda\vec{X}$\\
     &\\
     & then, $\brak{\vec{I}+\vec{A}}\vec{X}=\brak{1+\lambda}\vec{X}$\\
     &\\
     &Thus, $\vec{X}$ is an eigen vector of $\brak{\vec{A}+\vec{I}}$ corresponding to the eigen value $\brak{1+\lambda}$.\\
     
     &\\
     
     & Let $\lambda_1,\lambda_2$ be two eigen values of $\vec{A}$ and $\brak{1+\lambda_1},\brak{1+\lambda_2}$ be the eigen values of $\brak{\vec{A}+\vec{I}}.$\\
    &\\
    & $\implies$ Eigen value of $\vec{A}=\lambda_1,\lambda_2$\\
    &\\
    &  $\implies$ Eigen value of $\brak{\vec{A}+\vec{I}} = \lambda_1+1,\lambda_2+1$\\
    &\\
    \hline
    &\\
    & Since,\\
    &$\qquad\qquad \qquad \det\brak{\vec{A}+\vec{I}}=1+\det(\vec{A})$\\
    &\\
     & Trace of any matrix is sum of its eigen values. \\
    &\\
    & Determinant of matrix is product of its eigen values \\
    &\\
    & $\qquad\qquad\implies \brak{\lambda_1+1}\brak{\lambda_2+1}=1+\brak{\lambda_1\lambda_2}$\\
    &\\
    &$\qquad\qquad\implies\boxed{ \lambda_1+\lambda_2 = 0}$\\
    &\\
   
    &$\qquad\qquad\implies\boxed{ tr(\vec{A})=0}$\\
    &\\
    \hline
  	\multirow{3}{*}{Option 1 : $\det\vec{A}=0$ } 
	& \\
	& Let,\\
	& $\qquad\qquad\qquad\vec{A}=\myvec{0&-1\\0&0}$\\
	&\\
	& $\qquad\qquad\qquad\det{\vec{A}} = \mydet{0&-1\\0&0} = 0$\\
	&\\
	& $\qquad\qquad\qquad\brak{\vec{A}+\vec{I}}=\myvec{1&-1\\0&1}$\\
	&\\
	& $\qquad\qquad\qquad\det\brak{\vec{A}+\vec{I}}=\mydet{1&-1\\0&1}=1$\\
	&\\
	& $\qquad\qquad\qquad\implies\det\brak{\vec{A}+\vec{I}}=1+\det(\vec{A
	})$\\
	& Conclusion: {\begin{enumerate}
	\item $tr(\vec{A})=0$
	\item $\det{\vec{A}}= 0$
	\item $\vec{A} \neq \vec{0}$
	\item $\vec{A}$ is  singular.\end{enumerate}}\\
	\hline
	\multirow{3}{*}{Option 2 : $\vec{A}=0$} & \\
	& Let,\\
	& $\qquad\qquad\qquad\vec{A}= \myvec{0&0\\0&0}$\\
	&\\
    & $\qquad\qquad\qquad\det\vec{A}=0$\\
    &\\
    & $\qquad\qquad\qquad\det\brak{\vec{A}+\vec{I}}=1$\\
    &\\
    & $\qquad\qquad\implies\det\brak{\vec{A}+\vec{I}}=1+\det(\vec{A}) $\\
    
    \hline
   
    & Conclusion: {\begin{enumerate}
	\item $tr(\vec{A})=0$
	\item $\det{\vec{A}}=0$
	\item $\vec{A}=\vec{0}$
	\item $\vec{A}$ is singular.\end{enumerate}}\\
    \hline
	\multirow{3}{*}{Option 4: $\vec{A}$ is non singular}&\\
  	& Non Singular Matrix:
A non-singular matrix is a square one whose determinant is not zero.non-singular matrix is also  a full rank matrix.\\
&\\
	& Let,\\
	& $\qquad\qquad\qquad\vec{A}=\myvec{1&0\\0&-1}$\\
	&\\
	& $\qquad\qquad\qquad\det{\vec{A}} = \mydet{1&0\\0&-1} = -1$\\
	&\\
	& $\qquad\qquad\qquad\brak{\vec{A}+\vec{I}}=\myvec{2&0\\0&0}$\\
	&\\
	& $\qquad\qquad\qquad\det\brak{\vec{A}+\vec{I}}=\mydet{2&0\\0&0}=0$\\
	&\\
	& $\qquad\qquad\implies\det\brak{\vec{A}+\vec{I}}=1+\det(\vec{A})$\\
	&\\
	& Conclusion: {\begin{enumerate}
	\item $tr(\vec{A})=0$
	\item $\det{\vec{A}}\neq0$
	\item $\vec{A} \neq \vec{0}$
	\item $\vec{A}$ is non singular.\end{enumerate}}\\
	\hline
	\multirow{3}{*}{Conclusion}&\\
& In all options, $tr(\vec{A})=0$ satisfied.\\
&\\
& Thus, Option 3 is correct. \\
&\\

	\hline
	\caption{Solution Summary}
    \label{table:1}
\end{longtable}
\end{document}
