\documentclass[journal,12pt]{IEEEtran}
\usepackage{longtable}
\usepackage{setspace}
\usepackage{gensymb}
\singlespacing
\usepackage[cmex10]{amsmath}
\newcommand\myemptypage{
	\null
	\thispagestyle{empty}
	\addtocounter{page}{-1}
	\newpage
}
\usepackage{amsthm}
\usepackage{mdframed}
\usepackage{mathrsfs}
\usepackage{txfonts}
\usepackage{stfloats}
\usepackage{bm}
\usepackage{cite}
\usepackage{cases}
\usepackage{subfig}

\usepackage{longtable}
\usepackage{multirow}

\usepackage{enumitem}
\usepackage{mathtools}
\usepackage{steinmetz}
\usepackage{tikz}
\usepackage{circuitikz}
\usepackage{verbatim}
\usepackage{tfrupee}
\usepackage[breaklinks=true]{hyperref}
\usepackage{graphicx}
\usepackage{tkz-euclide}

\usetikzlibrary{calc,math}
\usepackage{listings}
    \usepackage{color}                                            %%
    \usepackage{array}                                            %%
    \usepackage{longtable}                                        %%
    \usepackage{calc}                                             %%
    \usepackage{multirow}                                         %%
    \usepackage{hhline}                                           %%
    \usepackage{ifthen}                                           %%
    \usepackage{lscape}     
\usepackage{multicol}
\usepackage{chngcntr}

\DeclareMathOperator*{\Res}{Res}

\renewcommand\thesection{\arabic{section}}
\renewcommand\thesubsection{\thesection.\arabic{subsection}}
\renewcommand\thesubsubsection{\thesubsection.\arabic{subsubsection}}

\renewcommand\thesectiondis{\arabic{section}}
\renewcommand\thesubsectiondis{\thesectiondis.\arabic{subsection}}
\renewcommand\thesubsubsectiondis{\thesubsectiondis.\arabic{subsubsection}}


\hyphenation{op-tical net-works semi-conduc-tor}
\def\inputGnumericTable{}                                 %%

\lstset{
%language=C,
frame=single, 
breaklines=true,
columns=fullflexible
}
\begin{document}
\onecolumn

\newtheorem{theorem}{Theorem}[section]
\newtheorem{problem}{Problem}
\newtheorem{proposition}{Proposition}[section]
\newtheorem{lemma}{Lemma}[section]
\newtheorem{corollary}[theorem]{Corollary}
\newtheorem{example}{Example}[section]
\newtheorem{definition}[problem]{Definition}

\newcommand{\BEQA}{\begin{eqnarray}}
\newcommand{\EEQA}{\end{eqnarray}}
\newcommand{\define}{\stackrel{\triangle}{=}}
\bibliographystyle{IEEEtran}
\raggedbottom
\setlength{\parindent}{0pt}
\providecommand{\mbf}{\mathbf}
\providecommand{\pr}[1]{\ensuremath{\Pr\left(#1\right)}}
\providecommand{\qfunc}[1]{\ensuremath{Q\left(#1\right)}}
\providecommand{\sbrak}[1]{\ensuremath{{}\left[#1\right]}}
\providecommand{\lsbrak}[1]{\ensuremath{{}\left[#1\right.}}
\providecommand{\rsbrak}[1]{\ensuremath{{}\left.#1\right]}}
\providecommand{\brak}[1]{\ensuremath{\left(#1\right)}}
\providecommand{\lbrak}[1]{\ensuremath{\left(#1\right.}}
\providecommand{\rbrak}[1]{\ensuremath{\left.#1\right)}}
\providecommand{\cbrak}[1]{\ensuremath{\left\{#1\right\}}}
\providecommand{\lcbrak}[1]{\ensuremath{\left\{#1\right.}}
\providecommand{\rcbrak}[1]{\ensuremath{\left.#1\right\}}}
\theoremstyle{remark}
\newtheorem{rem}{Remark}
\newcommand{\sgn}{\mathop{\mathrm{sgn}}}
\providecommand{\abs}[1]{\left\vert#1\right\vert}
\providecommand{\res}[1]{\Res\displaylimits_{#1}} 
\providecommand{\norm}[1]{\left\lVert#1\right\rVert}
%\providecommand{\norm}[1]{\lVert#1\rVert}
\providecommand{\mtx}[1]{\mathbf{#1}}
\providecommand{\mean}[1]{E\left[ #1 \right]}
\providecommand{\fourier}{\overset{\mathcal{F}}{ \rightleftharpoons}}
%\providecommand{\hilbert}{\overset{\mathcal{H}}{ \rightleftharpoons}}
\providecommand{\system}{\overset{\mathcal{H}}{ \longleftrightarrow}}
	%\newcommand{\solution}[2]{\textbf{Solution:}{#1}}
\newcommand{\solution}{\noindent \textbf{Solution: }}
\newcommand{\cosec}{\,\text{cosec}\,}
\providecommand{\dec}[2]{\ensuremath{\overset{#1}{\underset{#2}{\gtrless}}}}
\newcommand{\myvec}[1]{\ensuremath{\begin{pmatrix}#1\end{pmatrix}}}
\newcommand{\mydet}[1]{\ensuremath{\begin{vmatrix}#1\end{vmatrix}}}
\numberwithin{equation}{subsection}
\makeatletter
\@addtoreset{figure}{problem}
\makeatother
\let\StandardTheFigure\thefigure
\let\vec\mathbf
\renewcommand{\thefigure}{\theproblem}
\def\putbox#1#2#3{\makebox[0in][l]{\makebox[#1][l]{}\raisebox{\baselineskip}[0in][0in]{\raisebox{#2}[0in][0in]{#3}}}}
     \def\rightbox#1{\makebox[0in][r]{#1}}
     \def\centbox#1{\makebox[0in]{#1}}
     \def\topbox#1{\raisebox{-\baselineskip}[0in][0in]{#1}}
     \def\midbox#1{\raisebox{-0.5\baselineskip}[0in][0in]{#1}}
\vspace{3cm}
\title{Assignment 12}
\author{Neha Rani\\EE20MTECH14014}
\maketitle
\bigskip
\renewcommand{\thefigure}{\theenumi}
\renewcommand{\thetable}{\theenumi}
%
Download the latex-tikz codes from 
%
\begin{lstlisting}
https://github.com/neharani289/MatrixTheory/Assignment12
\end{lstlisting}
\section{\textbf{Problem}}
(hoffman/page189/5) : \\
%
Let
\begin{align}
\vec{A}=\myvec{6&-3&-2\\4&-1&-2\\10&-5&-3}
\end{align}
Is $A$ similar over the field $R$ to a diagonal matrix ? \\Is $A$ similar over the field C to a diagonal matrix?
%
%
%
\section{\textbf{Definition and Theorem used}}
\begin{longtable}{|l|l|}
\hline
\multirow{3}{*}{Theorem 2} & \\
& Let $T$ be the linear operator on a finite dimensional space $V$ and $c_1$,..,$c_k$ be a distinct \\
&\\
&characteristic values of $T$ and let $W_i$ be the null space of $(T-c_iI)$ then \\
&\\
& $1)T$ is diagonalizable \\
&\\
& $2)$ characteristic polynomial for $T$ is \\
& $f$ = $(x-c_1)^d^_1$....$(x-c_k)^d^_k$ and \\
&\\
& $3)$ $dim W_i$ = $d_i$, $i= 1,...k\\
&\\
\hline
\multirow{3}{*}{Condition} & \\
&
A linear operator $T$ on a $n$-dimensional space $V$ is\\ 
&\\ for diagonalization
& diagonalizable , if and only if $T$ has $n$ distinct \\
&\\
& characteristic vectors or null spaces corresponding to the characteristic values\\
\hline
\end{longtable}
\newpage
\section{\textbf{Solution}}
\begin{longtable}{|l|l|}
\hline
\multirow{3}{*}{Given} & \\
& Let the given matrix be  \\
&\\
& $\vec{A}$=$\myvec{6&-3&-2\\4&-1&-2\\10&-5&-3}$\\
&\\
\hline
\multirow{3}{*}{Finding Characteristics} & \\
&
Characteristics polynomial of the matrix $A$ is $det(xI-A)$\\ 
polynomial
& $\det(xI-A)$= $\left|
                \begin{array}{ccc}
                (x-6) & 3 & 2\\
                -4 & (x+1) & 2\\
                -10 & 5 & x+3
                \end{array} \right|$  \\
&\\
& Characteristic Polynomial = $(x-2)(x^2+1)=(x-2)(x-i)(x+i)$\\
&\\
\hline
\multirow{3}{*}{Checking whether $A$ similar} & \\
& As the characteristics  polynomial is not product of linear factors\\
over the field $R$ to a
& over $R$ . Therefore from Theorem 2, $A$ is not diagonalizable over $R$\\
diagonal matrix
&\\
\hline
\multirow{3}{*}{Checking whether $A$ similar} & \\
& The Characteristic Polynomial can be wriiten \\
$C$ over the field to a
& as a product of linear factors over $C$ i.e \\
diagonal matrix
&\\
& $\det(xI-A)=(x-2)(x-i)(x+i)$ \\
&\\
& To find characteristic values of the operator $\det(xI-A) = 0$ which gives  \\
& $c_1= 2 , c_2= i, c_3= -i$\\
&\\
& Thus over $C$ matrix $A$ has three distinct characteristic values.\\
&\\
&There will be atleast one characteristics vector i.e., one\\ & dimension with each characteristics value .\\
&\\
& From Theorem 2;\\
& $\sum_{i} W_i = n = 3$ , which is equal to $dim$ of $A$.\\
&\\
& Thus , $A$ is diagonalizable over $C$.\\
&\\
\hline
\multirow{3}{*}{Conclusion} & \\
& 1) $A$ is not diagonalizable over $R$. \\
&\\
& 2) $A$ is diagonalizable over $C$.\\
&\\
\hline
\end{longtable}
\end{document}