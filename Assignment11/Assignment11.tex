\documentclass[journal,12pt]{IEEEtran}
\usepackage{longtable}
\usepackage{setspace}
\usepackage{gensymb}
\singlespacing
\usepackage[cmex10]{amsmath}
\newcommand\myemptypage{
	\null
	\thispagestyle{empty}
	\addtocounter{page}{-1}
	\newpage
}
\usepackage{amsthm}
\usepackage{mdframed}
\usepackage{mathrsfs}
\usepackage{txfonts}
\usepackage{stfloats}
\usepackage{bm}
\usepackage{cite}
\usepackage{cases}
\usepackage{subfig}

\usepackage{longtable}
\usepackage{multirow}

\usepackage{enumitem}
\usepackage{mathtools}
\usepackage{steinmetz}
\usepackage{tikz}
\usepackage{circuitikz}
\usepackage{verbatim}
\usepackage{tfrupee}
\usepackage[breaklinks=true]{hyperref}
\usepackage{graphicx}
\usepackage{tkz-euclide}

\usetikzlibrary{calc,math}
\usepackage{listings}
    \usepackage{color}                                            %%
    \usepackage{array}                                            %%
    \usepackage{longtable}                                        %%
    \usepackage{calc}                                             %%
    \usepackage{multirow}                                         %%
    \usepackage{hhline}                                           %%
    \usepackage{ifthen}                                           %%
    \usepackage{lscape}     
\usepackage{multicol}
\usepackage{chngcntr}

\DeclareMathOperator*{\Res}{Res}

\renewcommand\thesection{\arabic{section}}
\renewcommand\thesubsection{\thesection.\arabic{subsection}}
\renewcommand\thesubsubsection{\thesubsection.\arabic{subsubsection}}

\renewcommand\thesectiondis{\arabic{section}}
\renewcommand\thesubsectiondis{\thesectiondis.\arabic{subsection}}
\renewcommand\thesubsubsectiondis{\thesubsectiondis.\arabic{subsubsection}}


\hyphenation{op-tical net-works semi-conduc-tor}
\def\inputGnumericTable{}                                 %%

\lstset{
%language=C,
frame=single, 
breaklines=true,
columns=fullflexible
}
\begin{document}
\onecolumn

\newtheorem{theorem}{Theorem}[section]
\newtheorem{problem}{Problem}
\newtheorem{proposition}{Proposition}[section]
\newtheorem{lemma}{Lemma}[section]
\newtheorem{corollary}[theorem]{Corollary}
\newtheorem{example}{Example}[section]
\newtheorem{definition}[problem]{Definition}

\newcommand{\BEQA}{\begin{eqnarray}}
\newcommand{\EEQA}{\end{eqnarray}}
\newcommand{\define}{\stackrel{\triangle}{=}}
\bibliographystyle{IEEEtran}
\raggedbottom
\setlength{\parindent}{0pt}
\providecommand{\mbf}{\mathbf}
\providecommand{\pr}[1]{\ensuremath{\Pr\left(#1\right)}}
\providecommand{\qfunc}[1]{\ensuremath{Q\left(#1\right)}}
\providecommand{\sbrak}[1]{\ensuremath{{}\left[#1\right]}}
\providecommand{\lsbrak}[1]{\ensuremath{{}\left[#1\right.}}
\providecommand{\rsbrak}[1]{\ensuremath{{}\left.#1\right]}}
\providecommand{\brak}[1]{\ensuremath{\left(#1\right)}}
\providecommand{\lbrak}[1]{\ensuremath{\left(#1\right.}}
\providecommand{\rbrak}[1]{\ensuremath{\left.#1\right)}}
\providecommand{\cbrak}[1]{\ensuremath{\left\{#1\right\}}}
\providecommand{\lcbrak}[1]{\ensuremath{\left\{#1\right.}}
\providecommand{\rcbrak}[1]{\ensuremath{\left.#1\right\}}}
\theoremstyle{remark}
\newtheorem{rem}{Remark}
\newcommand{\sgn}{\mathop{\mathrm{sgn}}}
\providecommand{\abs}[1]{\left\vert#1\right\vert}
\providecommand{\res}[1]{\Res\displaylimits_{#1}} 
\providecommand{\norm}[1]{\left\lVert#1\right\rVert}
%\providecommand{\norm}[1]{\lVert#1\rVert}
\providecommand{\mtx}[1]{\mathbf{#1}}
\providecommand{\mean}[1]{E\left[ #1 \right]}
\providecommand{\fourier}{\overset{\mathcal{F}}{ \rightleftharpoons}}
%\providecommand{\hilbert}{\overset{\mathcal{H}}{ \rightleftharpoons}}
\providecommand{\system}{\overset{\mathcal{H}}{ \longleftrightarrow}}
	%\newcommand{\solution}[2]{\textbf{Solution:}{#1}}
\newcommand{\solution}{\noindent \textbf{Solution: }}
\newcommand{\cosec}{\,\text{cosec}\,}
\providecommand{\dec}[2]{\ensuremath{\overset{#1}{\underset{#2}{\gtrless}}}}
\newcommand{\myvec}[1]{\ensuremath{\begin{pmatrix}#1\end{pmatrix}}}
\newcommand{\mydet}[1]{\ensuremath{\begin{vmatrix}#1\end{vmatrix}}}
\numberwithin{equation}{subsection}
\makeatletter
\@addtoreset{figure}{problem}
\makeatother
\let\StandardTheFigure\thefigure
\let\vec\mathbf
\renewcommand{\thefigure}{\theproblem}
\def\putbox#1#2#3{\makebox[0in][l]{\makebox[#1][l]{}\raisebox{\baselineskip}[0in][0in]{\raisebox{#2}[0in][0in]{#3}}}}
     \def\rightbox#1{\makebox[0in][r]{#1}}
     \def\centbox#1{\makebox[0in]{#1}}
     \def\topbox#1{\raisebox{-\baselineskip}[0in][0in]{#1}}
     \def\midbox#1{\raisebox{-0.5\baselineskip}[0in][0in]{#1}}
\vspace{3cm}
\title{Assignment 11}
\author{Neha Rani\\EE20MTECH14014}
\maketitle
\bigskip
\renewcommand{\thefigure}{\theenumi}
\renewcommand{\thetable}{\theenumi}
%
Download the latex-tikz codes from 
%
\begin{lstlisting}
https://github.com/neharani289/MatrixTheory/Assignment11
\end{lstlisting}
\section{\textbf{Problem}}
(UGC-june2017,71) : \\
%
Let $V$ be the vector space of polynomials of degree at most 3 in a variable $x$ with coefficients in $\mathbb{R}$. Let $T$=$d/dx$ be the linear transformation of $V$ to itself given by differentiation.\\

Which of the following are correct?\\
\begin{enumerate}
\item $T$ is invertible
\item $0$ is an eigenvalue of $T$
\item There is a basis with respect to which the matrix of $T$ is nilpotent.
\item The matrix of $T$ with respect to the basis $\{1,1+x,1+x+x^2,1+x+x^2+x^3\}$ is diagonal.
\end{enumerate}
%
%
%
\section{\textbf{Definition and Result used}}
\begin{longtable}{|l|l|}
\hline
\multirow{3}{*}{Nilpotent Matrix} 
& \\
& 1. If  all the eigen values of matrix is zero then it is said to nilpotent matrix \\
& 
2. Determinant and trace of nilpotent matrix are always zero.\\
\hline
\multirow{3}{*}{Invertible Matrix } & \\
&
A matrix is said to be invertible matrix if its determinant is non zero.\\
\hline
\multirow{3}{*}{Diagonal matrix} & \\
&
diagonal matrix is a matrix in which the entries outside the main diagonal are all zero.\\
\hline
\end{longtable}
\section{\textbf{Solution}}
\begin{longtable}{|l|l|}
\hline
\multirow{3}{*}{Given       } &\\
& $T$ : $P_3 \xrightarrow{} P_3$ \\ 
& \\
& $T:V\xrightarrow{}V$ be the linear operator given by differentiation wrt $x$\\
& $T(P(x)) \xrightarrow{}$ $P'(x)$ \\
& \\
& $A$ be the matrix of $T$ wrt some basis for $V$ \\
& Assume basis for $V$ be $\{1,x,x^2,x^3\}$ \\
\hline
\end{longtable}
\newpage
\begin{longtable}{|l|l|}
\hline
\multirow{3}{*}{Checking whether } &\\
& $T:V\rightarrow V$\\matrix of $T$ is nilpotent
& $TP(x) = P'(x)$\\
&Differentiating wrt $x$ to find matrix $A$;\\
& \qquad \qquad \qquad 
$T(1)$ = $0$ = $a_1+b_1x+c_1x^2+d_1x^3$\\
& \qquad \qquad \qquad 
$T(x)$= $1$ = $a_2+b_2x+c_2x^2+d_2x^3$\\
& \qquad \qquad \qquad 
$T(x^2)$ = $2x$ = $a_3+b_3x+c_3x^2+d_3x^3$\\
& \qquad \qquad \qquad 
$T(x^3)$ = $3x^2$ = $a_4+b_4x+c_4x^2+d_4x^3$\\
& 
Representing $A$ in matrix form ;\\
&
\qquad\qquad\qquad
$A$=$\myvec{0&1&0&0\\0&0&2&0\\0&0&0&3\\0&0&0&0}$\\
&
from the above matrix of $T$ we can say it is nilpotent matrix.\\
\hline
\multirow{3}{*}{ Checking eigen value of matrix $T$ } &\\
&
$A=\myvec{0-\lambda&1&0&0\\0&0-\lambda&2&0\\0&0&0-\lambda&3\\0&0&0&0-\lambda}$\\
&
$\implies \lambda=0$\\
&\\
\hline
\multirow{3}{*}{Checking whether matrix} & \\
&Since $\det{A}$= $0$. \\ of  $T$ is invertible 
&Therefore matrix of $T$ is not invertible \\
&\\
\hline
\multirow{3}{*}{Checking whether Matrix of $T$} & \\
& Let basis be $B'$ = $\{1,1+x,1+x+x^2,1+x+x^2+x^3\}$\\is diagonal matrix
& Differentiating wrt $x$ ;\\
&
$T(1) = 0 = a_1+b_1(1+x)+c_1(1+x+x^2)+d_1(1+x+x^2+x^3)$\\
&
$T(1+x)= 1 = a_2+b_2(1+x)+c_2(1+x+x^2)+d_2(1+x+x^2x^3)$\\
&
$T(1+x+x^2) = 1+2x = a_3+b_3(1+x)+c_3(1+x+x^2)$\\
&
\qquad\qquad\qquad+$d_3(1+x+x^2+x^3)$\\
&  
$T(1+x+x^2+x^3) = 1+2x+3x^2 = a_4+b_4(1+x)+c_4(1+x+x^2)$\\
&
\qquad\qquad\qquad\qquad+$d_4(1+x+x^2+x^3)$\\
&
\qquad\qquad\qquad $B=\myvec{0&1&-1&-1\\0&0&2&-1\\0&0&0&3\\0&0&0&0}$\\
&
above matrix is not a diagonal matrix\\
&\\
\hline
\multirow{3}{*}{Conclusion} & \\
&
Thus we can conclude \\
&
Option 2) and 3) are correct.\\
&\\
\hline
\end{longtable}
\end{document}
