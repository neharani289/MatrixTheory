\documentclass[journal,12pt]{IEEEtran}
\usepackage{longtable}
\usepackage{setspace}
\usepackage{gensymb}
\singlespacing
\usepackage[cmex10]{amsmath}
\newcommand\myemptypage{
	\null
	\thispagestyle{empty}
	\addtocounter{page}{-1}
	\newpage
}
\usepackage{amsthm}
\usepackage{mdframed}
\usepackage{mathrsfs}
\usepackage{txfonts}
\usepackage{stfloats}
\usepackage{bm}
\usepackage{cite}
\usepackage{cases}
\usepackage{subfig}

\usepackage{longtable}
\usepackage{multirow}

\usepackage{enumitem}
\usepackage{mathtools}
\usepackage{steinmetz}
\usepackage{tikz}
\usepackage{circuitikz}
\usepackage{verbatim}
\usepackage{tfrupee}
\usepackage[breaklinks=true]{hyperref}
\usepackage{graphicx}
\usepackage{tkz-euclide}

\usetikzlibrary{calc,math}
\usepackage{listings}
    \usepackage{color}                                            %%
    \usepackage{array}                                            %%
    \usepackage{longtable}                                        %%
    \usepackage{calc}                                             %%
    \usepackage{multirow}                                         %%
    \usepackage{hhline}                                           %%
    \usepackage{ifthen}                                           %%
    \usepackage{lscape}     
\usepackage{multicol}
\usepackage{chngcntr}

\DeclareMathOperator*{\Res}{Res}

\renewcommand\thesection{\arabic{section}}
\renewcommand\thesubsection{\thesection.\arabic{subsection}}
\renewcommand\thesubsubsection{\thesubsection.\arabic{subsubsection}}

\renewcommand\thesectiondis{\arabic{section}}
\renewcommand\thesubsectiondis{\thesectiondis.\arabic{subsection}}
\renewcommand\thesubsubsectiondis{\thesubsectiondis.\arabic{subsubsection}}


\hyphenation{op-tical net-works semi-conduc-tor}
\def\inputGnumericTable{}                                 %%

\lstset{
%language=C,
frame=single, 
breaklines=true,
columns=fullflexible
}
\begin{document}
\onecolumn

\newtheorem{theorem}{Theorem}[section]
\newtheorem{problem}{Problem}
\newtheorem{proposition}{Proposition}[section]
\newtheorem{lemma}{Lemma}[section]
\newtheorem{corollary}[theorem]{Corollary}
\newtheorem{example}{Example}[section]
\newtheorem{definition}[problem]{Definition}

\newcommand{\BEQA}{\begin{eqnarray}}
\newcommand{\EEQA}{\end{eqnarray}}
\newcommand{\define}{\stackrel{\triangle}{=}}
\bibliographystyle{IEEEtran}
\raggedbottom
\setlength{\parindent}{0pt}
\providecommand{\mbf}{\mathbf}
\providecommand{\pr}[1]{\ensuremath{\Pr\left(#1\right)}}
\providecommand{\qfunc}[1]{\ensuremath{Q\left(#1\right)}}
\providecommand{\sbrak}[1]{\ensuremath{{}\left[#1\right]}}
\providecommand{\lsbrak}[1]{\ensuremath{{}\left[#1\right.}}
\providecommand{\rsbrak}[1]{\ensuremath{{}\left.#1\right]}}
\providecommand{\brak}[1]{\ensuremath{\left(#1\right)}}
\providecommand{\lbrak}[1]{\ensuremath{\left(#1\right.}}
\providecommand{\rbrak}[1]{\ensuremath{\left.#1\right)}}
\providecommand{\cbrak}[1]{\ensuremath{\left\{#1\right\}}}
\providecommand{\lcbrak}[1]{\ensuremath{\left\{#1\right.}}
\providecommand{\rcbrak}[1]{\ensuremath{\left.#1\right\}}}
\theoremstyle{remark}
\newtheorem{rem}{Remark}
\newcommand{\sgn}{\mathop{\mathrm{sgn}}}
\providecommand{\abs}[1]{\left\vert#1\right\vert}
\providecommand{\res}[1]{\Res\displaylimits_{#1}} 
\providecommand{\norm}[1]{\left\lVert#1\right\rVert}
%\providecommand{\norm}[1]{\lVert#1\rVert}
\providecommand{\mtx}[1]{\mathbf{#1}}
\providecommand{\mean}[1]{E\left[ #1 \right]}
\providecommand{\fourier}{\overset{\mathcal{F}}{ \rightleftharpoons}}
%\providecommand{\hilbert}{\overset{\mathcal{H}}{ \rightleftharpoons}}
\providecommand{\system}{\overset{\mathcal{H}}{ \longleftrightarrow}}
	%\newcommand{\solution}[2]{\textbf{Solution:}{#1}}
\newcommand{\solution}{\noindent \textbf{Solution: }}
\newcommand{\cosec}{\,\text{cosec}\,}
\providecommand{\dec}[2]{\ensuremath{\overset{#1}{\underset{#2}{\gtrless}}}}
\newcommand{\myvec}[1]{\ensuremath{\begin{pmatrix}#1\end{pmatrix}}}
\newcommand{\mydet}[1]{\ensuremath{\begin{vmatrix}#1\end{vmatrix}}}
\numberwithin{equation}{subsection}
\makeatletter
\@addtoreset{figure}{problem}
\makeatother
\let\StandardTheFigure\thefigure
\let\vec\mathbf
\renewcommand{\thefigure}{\theproblem}
\def\putbox#1#2#3{\makebox[0in][l]{\makebox[#1][l]{}\raisebox{\baselineskip}[0in][0in]{\raisebox{#2}[0in][0in]{#3}}}}
     \def\rightbox#1{\makebox[0in][r]{#1}}
     \def\centbox#1{\makebox[0in]{#1}}
     \def\topbox#1{\raisebox{-\baselineskip}[0in][0in]{#1}}
     \def\midbox#1{\raisebox{-0.5\baselineskip}[0in][0in]{#1}}
\vspace{3cm}
\title{Assignment 14}
\author{Neha Rani\\EE20MTECH14014}
\maketitle
\renewcommand{\thefigure}{\theenumi}
\renewcommand{\thetable}{\theenumi}
%
Download the latex-tikz codes from 
%
\begin{lstlisting}
https://github.com/neharani289/MatrixTheory/Assignment14
\end{lstlisting}
\section{\textbf{Problem}}
(hoffman/page208/1b) : \\
%
%
Find an invertible matrix $\vec{P}$ such that $\vec{P^{-1}AP}$ and $\vec{P^{-1}BP}$ are both diagonal where $\vec{A}$ and $\vec{B}$ are real matrices.
\begin{align}
    \Vec{A}=\myvec{1&1\\1&1}\\ 
    \vec{B}=\myvec{1&a\\a&1}
\end{align}
\section{\textbf{Explanation}}
\renewcommand{\thetable}{1}
\begin{table}[ht!]
\centering
\begin{tabular}{|c|l|}
\hline
\multirow{3}{*}{Characteristic Polynomial} 
     & \\
     & For an $n\times n$ matrix $\vec{A}$, characteristic polynomial
     is defined by,\\
     &\\
     & $\qquad\qquad\qquad p\brak{x}=\mydet{x\Vec{I}-\Vec{A}}$\\
     &\\
\hline
\multirow{3}{*}{Theorem}
     &\\
     & According  to theorem 8, if a $2\times 2$ matrix has two 
     characteristics \\
     &\\
     & values then the $\vec{P}$ that diagonalize $\vec{A}$ will
     necessarily also   \\
     &\\
     & diagonalize any $\vec{B}$ that commutes with $\vec{A}$. \\
     &\\
\hline
\multirow{3}{*}{Basis}&\\
     & Let there exist a $\vec{P}$ in basis
     $\vec{\beta}=\{\vec{b}_1,.....,\vec{b}_n\}$ of $\mathbb{V}$  
     consisting of eigen vector\\
     &\\
     & which are common to both $\vec{A}$ and $\vec{B}$ such that\\
     &\\
     & $\qquad\qquad\qquad\vec{A}\vec{b}_i=\lambda_{i}\vec{b}_i
     \qquad\qquad\qquad\vec{B}\vec{b}_i=\mu_{i}\vec{b}_i$\\
     &\\
     & $\qquad\qquad\qquad\Lambda_{A}=\myvec{\lambda_1&0\\0&\lambda
     _2}\qquad\qquad\Lambda_{B}=\myvec{\mu_1&0\\0&\mu_2}$\\
     &\\
     & $\qquad\qquad\qquad\Lambda_{A}=\vec{P^{-1}AP} 
     \qquad\qquad\Lambda_{B}=\vec{P^{-1}BP}$\\
     &\\
\hline
\end{tabular}
\label{table:1}
\caption{Definitions and theorem used}
\end{table}
\newpage
\section{\textbf{Solution}}
\bigskip
\renewcommand{\thetable}{2}
\begin{longtable}{|c|l|l|}
\hline
\multirow{3}{*}
     &&\\
     Operations &  Matrix A &  Matrix B\\
\hline
     &&\\
     Characteristic Polynomial
     &$p\brak{x}=\mydet{x\Vec{I}-\Vec{A}}$
     &$p\brak{x}=\mydet{x\Vec{I}-\Vec{B}}$\\
     &&\\
     &$\qquad = \mydet{x-1&-1\\-1&x-1}$ 
     &$\qquad = \mydet{x-1&-a\\-a&x-1}$ \\
     &&\\
     &$\qquad=\brak{x-1}\brak{x-1}-1$
     &$\qquad=\brak{x-1}\brak{x-1}-a^{2}$\\
     &&\\
\hline
     &&\\
     Characteristic values
     &$p\brak{x}=0$
     &$p\brak{x}=0$\\
     &&\\
     &$x\brak{x-1}=0$
     &$\brak{x-1}^{2}-a^{2}=0$\\
     &&\\
     &$\lambda_1=0$ ,$\lambda_2=2$
     &$\mu_1=\brak{1-a}$ , $\mu_2=\brak{1+a}$\\ 
     &&\\
\hline
     &&\\
     Basis for Characteristics Values
     &$(\vec{A}-\lambda_i\vec{I})\vec{X}=0$     &$(\vec{B}-\mu_i\vec{I})\vec{X}=0$\\
     &&\\
     & $\implies$ For $\lambda_1 = 0$
     & $\implies$ For $\mu_1 = \brak{1-a}$\\
     &&\\
     & $\myvec{1&1\\1&1}\myvec{x_1\\x_2}= 0$
     & $\myvec{a&a\\a&a}\myvec{x_1\\x_2}= 0$\\
     &&\\
     & So, $\vec{b_1} = \myvec{-1\\1}$
     & So, $\vec{b_1} = \myvec{-1\\1}$\\
     &&\\
     & $\implies$ For $\lambda_2 = 2$
     & $\implies$ For $\mu_2 = \brak{1+a}$\\
     &&\\
     & $\myvec{-1&1\\1&-1}\myvec{x_1\\x_2}= 0$
     & $\myvec{-a&a\\a&-a}\myvec{x_1\\x_2}= 0$\\
     &&\\
     & So, $\vec{b_2} = \myvec{1\\1}$
     & So, $\vec{b_2} = \myvec{1\\1}$\\
     &&\\
\hline
     &&\\
     Invertible matrix 
     & Let $\vec{P} = \myvec{\vec{b_1} &\vec{b_2}}$
     & Let $\vec{P} = \myvec{\vec{b_1} &\vec{b_2}}$\\
     &&\\
     & Then,
     & Then,\\
     &&\\
     & $\vec{P}=\myvec{-1&1\\1&1}$
     & $ \implies \vec{P}=\myvec{-1&1\\1&1}$\\
     &&\\
\hline
     &&\\
     Verification
     & $\Lambda_A=\myvec{0&0\\0&2}$ &
     $\Lambda_B=\myvec{1-a&0\\0&1+a}$\\
     &&\\
     & $\Lambda_{A}=\vec{P^{-1}AP}$&$\Lambda_{B}=\vec{P^{-1}BP}$\\
     &&\\
     & $\Lambda_A=\myvec{\frac{-1}{2}&\frac{1}{2}\\ \frac 
     {1}{2}&\frac{1}{2}} \myvec{1&1\\1&1} \myvec{-1&1\\1&1}$
     & $\Lambda_B=\myvec{\frac{-1}{2}&\frac{1}{2}\\ \frac
     {1}{2}&\frac{1}{2}} \myvec{1&a\\a&1} \myvec{-1&1\\1&1}$\\
     &&\\
     & $=\myvec{0&0\\0&2}=\Lambda_A$ 
     & $=\myvec{1-a&0\\0&1+a}=\Lambda_B$\\
     &&\\
     \hline
     &&\\
     Conclusion
     &  $\vec{P}=\myvec{-1&1\\1&1}$
     &  $\vec{P}=\myvec{-1&1\\1&1}$\\
     &&\\
\hline
\caption{Finding an Invertible matrix}
\label{table:2}
\end{longtable}
\end{document}
