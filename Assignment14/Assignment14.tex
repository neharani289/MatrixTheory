\documentclass[journal,12pt]{IEEEtran}
\usepackage{longtable}
\usepackage{setspace}
\usepackage{gensymb}
\singlespacing
\usepackage[cmex10]{amsmath}
\newcommand\myemptypage{
	\null
	\thispagestyle{empty}
	\addtocounter{page}{-1}
	\newpage
}
\usepackage{amsthm}
\usepackage{mdframed}
\usepackage{mathrsfs}
\usepackage{txfonts}
\usepackage{stfloats}
\usepackage{bm}
\usepackage{cite}
\usepackage{cases}
\usepackage{subfig}

\usepackage{longtable}
\usepackage{multirow}

\usepackage{enumitem}
\usepackage{mathtools}
\usepackage{steinmetz}
\usepackage{tikz}
\usepackage{circuitikz}
\usepackage{verbatim}
\usepackage{tfrupee}
\usepackage[breaklinks=true]{hyperref}
\usepackage{graphicx}
\usepackage{tkz-euclide}

\usetikzlibrary{calc,math}
\usepackage{listings}
    \usepackage{color}                                            %%
    \usepackage{array}                                            %%
    \usepackage{longtable}                                        %%
    \usepackage{calc}                                             %%
    \usepackage{multirow}                                         %%
    \usepackage{hhline}                                           %%
    \usepackage{ifthen}                                           %%
    \usepackage{lscape}     
\usepackage{multicol}
\usepackage{chngcntr}

\DeclareMathOperator*{\Res}{Res}

\renewcommand\thesection{\arabic{section}}
\renewcommand\thesubsection{\thesection.\arabic{subsection}}
\renewcommand\thesubsubsection{\thesubsection.\arabic{subsubsection}}

\renewcommand\thesectiondis{\arabic{section}}
\renewcommand\thesubsectiondis{\thesectiondis.\arabic{subsection}}
\renewcommand\thesubsubsectiondis{\thesubsectiondis.\arabic{subsubsection}}


\hyphenation{op-tical net-works semi-conduc-tor}
\def\inputGnumericTable{}                                 %%

\lstset{
%language=C,
frame=single, 
breaklines=true,
columns=fullflexible
}
\begin{document}
\onecolumn

\newtheorem{theorem}{Theorem}[section]
\newtheorem{problem}{Problem}
\newtheorem{proposition}{Proposition}[section]
\newtheorem{lemma}{Lemma}[section]
\newtheorem{corollary}[theorem]{Corollary}
\newtheorem{example}{Example}[section]
\newtheorem{definition}[problem]{Definition}

\newcommand{\BEQA}{\begin{eqnarray}}
\newcommand{\EEQA}{\end{eqnarray}}
\newcommand{\define}{\stackrel{\triangle}{=}}
\bibliographystyle{IEEEtran}
\raggedbottom
\setlength{\parindent}{0pt}
\providecommand{\mbf}{\mathbf}
\providecommand{\pr}[1]{\ensuremath{\Pr\left(#1\right)}}
\providecommand{\qfunc}[1]{\ensuremath{Q\left(#1\right)}}
\providecommand{\sbrak}[1]{\ensuremath{{}\left[#1\right]}}
\providecommand{\lsbrak}[1]{\ensuremath{{}\left[#1\right.}}
\providecommand{\rsbrak}[1]{\ensuremath{{}\left.#1\right]}}
\providecommand{\brak}[1]{\ensuremath{\left(#1\right)}}
\providecommand{\lbrak}[1]{\ensuremath{\left(#1\right.}}
\providecommand{\rbrak}[1]{\ensuremath{\left.#1\right)}}
\providecommand{\cbrak}[1]{\ensuremath{\left\{#1\right\}}}
\providecommand{\lcbrak}[1]{\ensuremath{\left\{#1\right.}}
\providecommand{\rcbrak}[1]{\ensuremath{\left.#1\right\}}}
\theoremstyle{remark}
\newtheorem{rem}{Remark}
\newcommand{\sgn}{\mathop{\mathrm{sgn}}}
\providecommand{\abs}[1]{\left\vert#1\right\vert}
\providecommand{\res}[1]{\Res\displaylimits_{#1}} 
\providecommand{\norm}[1]{\left\lVert#1\right\rVert}
%\providecommand{\norm}[1]{\lVert#1\rVert}
\providecommand{\mtx}[1]{\mathbf{#1}}
\providecommand{\mean}[1]{E\left[ #1 \right]}
\providecommand{\fourier}{\overset{\mathcal{F}}{ \rightleftharpoons}}
%\providecommand{\hilbert}{\overset{\mathcal{H}}{ \rightleftharpoons}}
\providecommand{\system}{\overset{\mathcal{H}}{ \longleftrightarrow}}
	%\newcommand{\solution}[2]{\textbf{Solution:}{#1}}
\newcommand{\solution}{\noindent \textbf{Solution: }}
\newcommand{\cosec}{\,\text{cosec}\,}
\providecommand{\dec}[2]{\ensuremath{\overset{#1}{\underset{#2}{\gtrless}}}}
\newcommand{\myvec}[1]{\ensuremath{\begin{pmatrix}#1\end{pmatrix}}}
\newcommand{\mydet}[1]{\ensuremath{\begin{vmatrix}#1\end{vmatrix}}}
\numberwithin{equation}{subsection}
\makeatletter
\@addtoreset{figure}{problem}
\makeatother
\let\StandardTheFigure\thefigure
\let\vec\mathbf
\renewcommand{\thefigure}{\theproblem}
\def\putbox#1#2#3{\makebox[0in][l]{\makebox[#1][l]{}\raisebox{\baselineskip}[0in][0in]{\raisebox{#2}[0in][0in]{#3}}}}
     \def\rightbox#1{\makebox[0in][r]{#1}}
     \def\centbox#1{\makebox[0in]{#1}}
     \def\topbox#1{\raisebox{-\baselineskip}[0in][0in]{#1}}
     \def\midbox#1{\raisebox{-0.5\baselineskip}[0in][0in]{#1}}
\vspace{3cm}
\title{Assignment 14}
\author{Neha Rani\\EE20MTECH14014}
\maketitle
\bigskip
\renewcommand{\thefigure}{\theenumi}
\renewcommand{\thetable}{\theenumi}
%
Download the latex-tikz codes from 
%
\begin{lstlisting}
https://github.com/neharani289/MatrixTheory/Assignment14
\end{lstlisting}
\section{\textbf{Problem}}
(hoffman/page208/1b) : \\
%
Find an invertible matrix $\vec{P}$ such that $\vec{P^{-1}AP}$ and $\vec{P^{-1}BP}$ are both diagonal where $\vec{A}$ and $\vec{B}$ are real matrices.

\begin{align}
    \Vec{A}=\myvec{1&1\\1&1}\\ 
   \nonumber \\ 
    \vec{B}=\myvec{1&a\\a&1}
\end{align}
\bigskip\\
\section{\textbf{Explanation}}
\renewcommand{\thetable}{1}
\begin{table}[ht!]
\centering
\begin{tabular}{|c|l|}
    \hline
	\multirow{3}{*}{Characteristic Polynomial} 
	& \\
	& For an $n\times n$ matrix $\vec{A}$, characteristic polynomial is defined by,\\
	&\\
	& $\qquad\qquad\qquad p\brak{x}=\mydet{x\Vec{I}-\Vec{A}}$\\
	&\\
	\hline
	\multirow{3}{*}{Theorem}
    &\\
    & Acc to theorem 8, if a $2\times 2$ matrix has two characteristics values then the \\
    &\\
    & $\vec{P}$ that diagonalize $\vec{A}$ will
    necessarily also diagonalize any $\vec{B}$ \\
    &\\
    & that commutes with $\vec{A}$. \\
    &\\
    \hline
\end{tabular}
\label{table:1}
    \caption{Definitions and theorem used}
\end{table}
\newpage
\section{\textbf{Solution}}
\renewcommand{\thetable}{2}
\begin{longtable}{|c|l|}
    \hline
	\multirow{3}{*}{Characteristic polynomial} 
	& \\
	& $p\brak{x}=\mydet{x\Vec{I}-\Vec{A}}$\\
	& $\qquad = \mydet{x-1&1\\1&x-1}$\\
	& $\qquad=\brak{x-1}\brak{x-1}-1$\\
	&$\qquad=x^2-2x$\\
	&$\qquad=x\brak{x-2}$\\
	&\\
	\hline
	\multirow{3}{*}{Characteristic values} 
	&\\
	& $p\brak{x}=0$\\
	& $\implies x\brak{x-2}=0$\\
	&  $\implies c_1=0,c_2=2$\\
	&\\
	\hline
	\multirow{3}{*}{Basis for characteristic values} & \\
	& Basis for Characteristics value $c_1=0$ will be\\ & obtained by solving homogenous equation $\brak{\vec{A}-c_1\vec{I}}x=0$  \\
	&\\
	& $\implies \myvec{1&1\\1&1}x = 0$\\
	& After solving Basis for characteristics value $c_1$ is $\vec{B_1} = \myvec{-1\\1}$ \\
	& Similarly for $c_2$ we get;\\
	& $\vec{B_2} = \myvec{1\\1}$\\
	&\\
	& $c_1 : \myvec{1&1\\1&1}\myvec{-1\\1} = \myvec{0\\0}$\\
	&\\
	&$c_2 : \myvec{-1&1\\1&-1}\myvec{1\\1} = \myvec{0\\0}$\\
	&\\
	\hline
	\multirow{3}{*}{Invertible matrix} & \\
	&  Now, Invertible matrix $\vec{P}$ is given by\\
	&\\
	& So, $\vec{P} = \myvec{-1&1\\1&1}$\\
	&\\
	& $\vec{P^{-1}} = \myvec{\frac{-1}{2}&\frac{1}{2}\\ \frac {1}{2}&\frac{1}{2}} $\\
	&\\
	& $\implies \vec{P^{-1}AP}=\myvec{\frac{-1}{2}&\frac{1}{2}\\ \frac {1}{2}&\frac{1}{2}} \myvec{1&1\\1&1} \myvec{-1&1\\1&1} = \myvec{0&0\\0&2}$\\
	&\\
	& $\implies \vec{P^{-1}BP}=\myvec{\frac{-1}{2}&\frac{1}{2}\\ \frac {1}{2}&\frac{1}{2}} \myvec{1&a\\a&1} \myvec{-1&1\\1&1} = \myvec{1-a&0\\0&1-a}$\\
	&\\
	\hline
	\caption{Solution Summary}
    \label{table:2}
\end{longtable}
\end{document}