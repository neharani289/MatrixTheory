
\documentclass[journal,12pt,twocolumn]{IEEEtran}
%
\usepackage{setspace}
\usepackage{gensymb}
\singlespacing
\usepackage[cmex10]{amsmath}
\usepackage{amsthm}
\usepackage{mathrsfs}
\usepackage{txfonts}
\usepackage{stfloats}
\usepackage{bm}
\usepackage{cite}
\usepackage{cases}
\usepackage{subfig}
\usepackage{longtable}
\usepackage{multirow}
%\usepackage{algorithm}
\usepackage{enumitem}
\usepackage{mathtools}
\usepackage{steinmetz}
\usepackage{tikz}
\usepackage{circuitikz}
\usepackage{verbatim}
\usepackage{tfrupee}
\usepackage[breaklinks=true]{hyperref}
%\usepackage{stmaryrd}
\usepackage{tkz-euclide} % loads  TikZ and tkz-base
%\usetkzobj{all}
\usetikzlibrary{calc,math}
\usepackage{listings}
\usepackage{color}                                            %%
\usepackage{array}                                            %%
\usepackage{longtable}                                        %%
\usepackage{calc}                                             %%
\usepackage{multirow}                                         %%
\usepackage{hhline}                                           %%
\usepackage{ifthen}                                           %%
%optionally (for landscape tables embedded in another document): %%
\usepackage{lscape}     
\usepackage{multicol}
\usepackage{chngcntr}
%\usepackage{enumerate}

%\usepackage{wasysym}
%\newcounter{MYtempeqncnt}
\DeclareMathOperator*{\Res}{Res}
%\renewcommand{\baselinestretch}{2}
\renewcommand\thesection{\arabic{section}}
\renewcommand\thesubsection{\thesection.\arabic{subsection}}
\renewcommand\thesubsubsection{\thesubsection.\arabic{subsubsection}}
\newcommand\numberthis{\addtocounter{equation}{1}\tag{\theequation}}
\renewcommand\thesectiondis{\arabic{section}}
\renewcommand\thesubsectiondis{\thesectiondis.\arabic{subsection}}
\renewcommand\thesubsubsectiondis{\thesubsectiondis.\arabic{subsubsection}}

% correct bad hyphenation here
\hyphenation{op-tical net-works semi-conduc-tor}
\def\inputGnumericTable{}                                 %%

\lstset{
	%language=C,
	frame=single, 
	breaklines=true,
	columns=fullflexible
}


\begin{document}
	%
	
	
	\newtheorem{theorem}{Theorem}[section]
	\newtheorem{problem}{Problem}
	\newtheorem{proposition}{Proposition}[section]
	\newtheorem{lemma}{Lemma}[section]
	\newtheorem{corollary}[theorem]{Corollary}
	\newtheorem{example}{Example}[section]
	\newtheorem{definition}[problem]{Definition}
	
	\newcommand{\BEQA}{\begin{eqnarray}}
	\newcommand{\EEQA}{\end{eqnarray}}
	\newcommand{\define}{\stackrel{\triangle}{=}}
	\bibliographystyle{IEEEtran}
	%\bibliographystyle{ieeetr}
	\providecommand{\mbf}{\mathbf}
	\providecommand{\pr}[1]{\ensuremath{\Pr\left(#1\right)}}
	\providecommand{\qfunc}[1]{\ensuremath{Q\left(#1\right)}}
	\providecommand{\sbrak}[1]{\ensuremath{{}\left[#1\right]}}
	\providecommand{\lsbrak}[1]{\ensuremath{{}\left[#1\right.}}
	\providecommand{\rsbrak}[1]{\ensuremath{{}\left.#1\right]}}
	\providecommand{\brak}[1]{\ensuremath{\left(#1\right)}}
	\providecommand{\lbrak}[1]{\ensuremath{\left(#1\right.}}
	\providecommand{\rbrak}[1]{\ensuremath{\left.#1\right)}}
	\providecommand{\cbrak}[1]{\ensuremath{\left\{#1\right\}}}
	\providecommand{\lcbrak}[1]{\ensuremath{\left\{#1\right.}}
	\providecommand{\rcbrak}[1]{\ensuremath{\left.#1\right\}}}
	\theoremstyle{remark}
	\newtheorem{rem}{Remark}
	\newcommand{\sgn}{\mathop{\mathrm{sgn}}}
	\providecommand{\abs}[1]{\left\vert#1\right\vert}
	\providecommand{\res}[1]{\Res\displaylimits_{#1}} 
	\providecommand{\norm}[1]{\left\lVert#1\right\rVert}
	%\providecommand{\norm}[1]{\lVert#1\rVert}
	\providecommand{\mtx}[1]{\mathbf{#1}}
	\providecommand{\mean}[1]{E\left[ #1 \right]}
	\providecommand{\fourier}{\overset{\mathcal{F}}{ \rightleftharpoons}}
	%\providecommand{\hilbert}{\overset{\mathcal{H}}{ \rightleftharpoons}}
	\providecommand{\system}{\overset{\mathcal{H}}{ \longleftrightarrow}}
	%\newcommand{\solution}[2]{\textbf{Solution:}{#1}}
	\newcommand{\solution}{\noindent \textbf{Solution: }}
	\newcommand{\cosec}{\,\text{cosec}\,}
	\providecommand{\dec}[2]{\ensuremath{\overset{#1}{\underset{#2}{\gtrless}}}}
	\newcommand{\myvec}[1]{\ensuremath{\begin{pmatrix}#1\end{pmatrix}}}
	\newcommand{\mydet}[1]{\ensuremath{\begin{vmatrix}#1\end{vmatrix}}}
	\numberwithin{equation}{subsection}
	\makeatletter
	\@addtoreset{figure}{problem}
	\makeatother
	\let\StandardTheFigure\thefigure
	\let\vec\mathbf
	\renewcommand{\thefigure}{\theproblem}
	\def\putbox#1#2#3{\makebox[0in][l]{\makebox[#1][l]{}\raisebox{\baselineskip}[0in][0in]{\raisebox{#2}[0in][0in]{#3}}}}
	\def\rightbox#1{\makebox[0in][r]{#1}}
	\def\centbox#1{\makebox[0in]{#1}}
	\def\topbox#1{\raisebox{-\baselineskip}[0in][0in]{#1}}
	\def\midbox#1{\raisebox{-0.5\baselineskip}[0in][0in]{#1}}
	\vspace{3cm}
	\title{Assignment 10}
	\author{Neha Rani \\ EE20MTECH14014}
	
	
	\maketitle
	\newpage
	%\tableofcontents
	\bigskip
	\renewcommand{\thefigure}{\theenumi}
	\renewcommand{\thetable}{\theenumi}
	\counterwithout{figure}{section}
	\counterwithout{figure}{subsection}
	\date{Today}
\begin{abstract}
This document solves a problem based on Linear Transformation .
\end{abstract}
%
Download latex-tikz codes from 
%
\begin{lstlisting}
https://github.com/neharani289/MatrixTheory/blob/master/Assignment10
\end{lstlisting}
\section{\textbf{Problem}}
Let \textbf{T} be the linear transformation from  $\mathbb{R}^3$ into $\mathbb{R}^2$ defined by
\begin{align}
 \textbf{T} \brak{x_1, x_2,x_3} = \brak{x_1+x_2,2x_3-x_1}\label{T}
 \end{align}
\intertext{If} $\beta$ & = \brak{\alpha_1,\alpha_2,\alpha_3} 
\intertext {and} $\beta'$ &= \brak{\beta_1,\beta_2}
where 
\begin{align}
\vec{\alpha_1}  = \brak{1,0,-1},  \vec{\alpha_2} = \brak{1,1,1}, \vec{\alpha_3} = \brak{1,0,0}\nonumber\end{align}
\begin{align}
\vec{\beta_1} = \brak{0,1}, \vec{\beta_2} = \brak{1,0}\nonumber
\end{align}
What is the matrix of \textbf{T} relative to the pair $\beta$, $\beta'$
\section{\textbf{Solution}}
Let
\begin{align}
\beta &= \brak{\alpha_1,\alpha_2,\alpha_3}
\end{align}
\begin{align}
\implies \beta=\myvec{1&1&1\\0&1&0\\-1&1&0}
\end{align}
and
\begin{align}
\beta' = \{\beta_1 , \beta_2\}
\end{align}
\begin{align}
\implies \beta'=\myvec{0&1\\1&0}
\end{align}
T is defined by
\begin{align}
   T\brak{\vec{x}} = \vec{A}\vec{x}
 \end{align}
using equation \eqref{T}
\begin{align}
T\myvec{x_1\\x_2\\x_3}=\myvec{x_1+x_2\\2x_3-x_1}
\end{align}
R.H.S of the equation can be written as a product of 2$\times$3 and 3$\times$1 matrices,
\begin{align}
=\myvec{1&1&0\\-1&0&2}\myvec{x_1\\x_2\\x_3}\\
\implies \vec{A}=\myvec{1&1&0\\-1&0&2}
\end{align}
Now,
 \begin{align}
    T\brak{\beta}=\myvec{1&1&0\\-1&0&2}{\beta}
\end{align}
\begin{align}
    T\brak{\beta}=\myvec{1&1&0\\-1&0&2}\myvec{1&1&1\\0&1&0\\-1&1&0}
    =\myvec{1&2&1\\-3&1&-1}
\end{align}
To find relative matrix we will use row reduce augmented matrix.
\begin{align}
    \myvec{1&2&1&\vline&0&1\\-3&1&-1&\vline&1&0}
\end{align}
\begin{align}
\myvec{1&2&1&\vline&0&1\\-3&1&-1&\vline&1&0}\xleftrightarrow{R_1\leftrightarrow R_2}\myvec{-3&1&-1&\vline&1&0\\1&2&1&\vline&0&1}
\end{align}
Hence the matrix of \textbf{T} in the order basis of $\beta^{'}$
\begin{align}
    \vec{B}=\myvec{-3&1&-1\\1&2&1}
\end{align}
 Therefore matrix of relative to the pair $\beta$ , $\beta^{'}$  
 \begin{align}
  T\brak{\vec{\beta}}=\vec{A}\beta = \vec{B}\beta^{'}=\myvec{-3&1&-1\\1&2&1}\vec{\beta^{'}}   
 \end{align}
\end{document}
