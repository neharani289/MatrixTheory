\documentclass[journal,12pt]{IEEEtran}
\usepackage{longtable}
\usepackage{setspace}
\usepackage{gensymb}
\singlespacing
\usepackage[cmex10]{amsmath}
\newcommand\myemptypage{
	\null
	\thispagestyle{empty}
	\addtocounter{page}{-1}
	\newpage
}
\usepackage{amsthm}
\usepackage{mdframed}
\usepackage{mathrsfs}
\usepackage{txfonts}
\usepackage{stfloats}
\usepackage{bm}
\usepackage{cite}
\usepackage{cases}
\usepackage{subfig}

\usepackage{longtable}
\usepackage{multirow}

\usepackage{enumitem}
\usepackage{mathtools}
\usepackage{steinmetz}
\usepackage{tikz}
\usepackage{circuitikz}
\usepackage{verbatim}
\usepackage{tfrupee}
\usepackage[breaklinks=true]{hyperref}
\usepackage{graphicx}
\usepackage{tkz-euclide}

\usetikzlibrary{calc,math}
\usepackage{listings}
    \usepackage{color}                                            %%
    \usepackage{array}                                            %%
    \usepackage{longtable}                                        %%
    \usepackage{calc}                                             %%
    \usepackage{multirow}                                         %%
    \usepackage{hhline}                                           %%
    \usepackage{ifthen}                                           %%
    \usepackage{lscape}     
\usepackage{multicol}
\usepackage{chngcntr}

\DeclareMathOperator*{\Res}{Res}

\renewcommand\thesection{\arabic{section}}
\renewcommand\thesubsection{\thesection.\arabic{subsection}}
\renewcommand\thesubsubsection{\thesubsection.\arabic{subsubsection}}

\renewcommand\thesectiondis{\arabic{section}}
\renewcommand\thesubsectiondis{\thesectiondis.\arabic{subsection}}
\renewcommand\thesubsubsectiondis{\thesubsectiondis.\arabic{subsubsection}}


\hyphenation{op-tical net-works semi-conduc-tor}
\def\inputGnumericTable{}                                 %%

\lstset{
%language=C,
frame=single, 
breaklines=true,
columns=fullflexible
}
\begin{document}
\onecolumn

\newtheorem{theorem}{Theorem}[section]
\newtheorem{problem}{Problem}
\newtheorem{proposition}{Proposition}[section]
\newtheorem{lemma}{Lemma}[section]
\newtheorem{corollary}[theorem]{Corollary}
\newtheorem{example}{Example}[section]
\newtheorem{definition}[problem]{Definition}

\newcommand{\BEQA}{\begin{eqnarray}}
\newcommand{\EEQA}{\end{eqnarray}}
\newcommand{\define}{\stackrel{\triangle}{=}}
\bibliographystyle{IEEEtran}
\raggedbottom
\setlength{\parindent}{0pt}
\providecommand{\mbf}{\mathbf}
\providecommand{\pr}[1]{\ensuremath{\Pr\left(#1\right)}}
\providecommand{\qfunc}[1]{\ensuremath{Q\left(#1\right)}}
\providecommand{\sbrak}[1]{\ensuremath{{}\left[#1\right]}}
\providecommand{\lsbrak}[1]{\ensuremath{{}\left[#1\right.}}
\providecommand{\rsbrak}[1]{\ensuremath{{}\left.#1\right]}}
\providecommand{\brak}[1]{\ensuremath{\left(#1\right)}}
\providecommand{\lbrak}[1]{\ensuremath{\left(#1\right.}}
\providecommand{\rbrak}[1]{\ensuremath{\left.#1\right)}}
\providecommand{\cbrak}[1]{\ensuremath{\left\{#1\right\}}}
\providecommand{\lcbrak}[1]{\ensuremath{\left\{#1\right.}}
\providecommand{\rcbrak}[1]{\ensuremath{\left.#1\right\}}}
\theoremstyle{remark}
\newtheorem{rem}{Remark}
\newcommand{\sgn}{\mathop{\mathrm{sgn}}}
\providecommand{\abs}[1]{\left\vert#1\right\vert}
\providecommand{\res}[1]{\Res\displaylimits_{#1}} 
\providecommand{\norm}[1]{\left\lVert#1\right\rVert}
%\providecommand{\norm}[1]{\lVert#1\rVert}
\providecommand{\mtx}[1]{\mathbf{#1}}
\providecommand{\mean}[1]{E\left[ #1 \right]}
\providecommand{\fourier}{\overset{\mathcal{F}}{ \rightleftharpoons}}
%\providecommand{\hilbert}{\overset{\mathcal{H}}{ \rightleftharpoons}}
\providecommand{\system}{\overset{\mathcal{H}}{ \longleftrightarrow}}
	%\newcommand{\solution}[2]{\textbf{Solution:}{#1}}
\newcommand{\solution}{\noindent \textbf{Solution: }}
\newcommand{\cosec}{\,\text{cosec}\,}
\providecommand{\dec}[2]{\ensuremath{\overset{#1}{\underset{#2}{\gtrless}}}}
\newcommand{\myvec}[1]{\ensuremath{\begin{pmatrix}#1\end{pmatrix}}}
\newcommand{\mydet}[1]{\ensuremath{\begin{vmatrix}#1\end{vmatrix}}}
\numberwithin{equation}{subsection}
\makeatletter
\@addtoreset{figure}{problem}
\makeatother
\let\StandardTheFigure\thefigure
\let\vec\mathbf
\renewcommand{\thefigure}{\theproblem}
\def\putbox#1#2#3{\makebox[0in][l]{\makebox[#1][l]{}\raisebox{\baselineskip}[0in][0in]{\raisebox{#2}[0in][0in]{#3}}}}
     \def\rightbox#1{\makebox[0in][r]{#1}}
     \def\centbox#1{\makebox[0in]{#1}}
     \def\topbox#1{\raisebox{-\baselineskip}[0in][0in]{#1}}
     \def\midbox#1{\raisebox{-0.5\baselineskip}[0in][0in]{#1}}
\vspace{3cm}
\title{Assignment 15}
\author{Neha Rani\\EE20MTECH14014}
\maketitle
\bigskip
\renewcommand{\thefigure}{\theenumi}
\renewcommand{\thetable}{\theenumi}
%
Download the latex-tikz codes from 
%
\begin{lstlisting}
https://github.com/neharani289/MatrixTheory/Assignment15
\end{lstlisting}
\section{\textbf{Problem}}
%
(hoffman/page213/3) : 
\bigskip\\
%
Find a projection $\vec{E}$ which projects $\mathbb{R}^{2}$ onto the subspace spanned by $\brak{1,-1}$ along the subspace spanned by $\brak{1,2}$.
\bigskip
\section{\textbf{Solution}}
\renewcommand{\thetable}{1}
\begin{longtable}{|p{4cm}|p{14cm}|}
\hline
\multirow{3}{*}{Given} 
    	& \\
     	&  Let $\myvec{x\\y} \in \mathbb{R}^{2}$\\
     	&{\begin{align}\label{eq1}
     	& \myvec{x\\y} = a\myvec{1\\-1} +b\myvec{1\\2}
     	\end{align}}\\
     	& where $\myvec{a\\b}$ is representation of $\myvec{x\\y}$ in new basis.\\
     	&\\
\hline
\multirow{3}{*}{To find} 
     	&\\
        &{\begin{align}\label{eq2}
        \vec{E}\myvec{x\\y} = a\myvec{1\\-1}
        \end{align}}\\
     	
\hline
\multirow{3}{*}{Finding a Projection $\vec{E}$} & \\
       & We know in standard order basis ;\\
       & {\begin{align}\label{eq3}
       \myvec{x\\y} : \myvec{1\\0}x + \myvec{0\\1}y
       \end{align}}\\
       &\\
       & Express $\myvec{1\\0},\myvec{0\\1}$ in the basis       $\myvec{1\\-1},\myvec{1\\2}$\\
      &\\
      & {\begin{align}\label{eq4}
      \myvec{1\\0} = p\myvec{1\\-1} + q\myvec{1\\2}
      \end{align}}\\
      & where $\myvec{p\\q}$ is representation of $\myvec{1\\0}$ in  the new basis.\\
      &{\begin{align}\label{eq5}
      &\implies \myvec{p\\q}=\myvec{\frac{2}{3}\\\frac{1}{3}}
      &\end{align}}\\
      & similarly;\\
      & {\begin{align}\label{eq6} 
      \myvec{0\\1}= r\myvec{1\\-1} +s \myvec{1\\2} 
      \end{align}}\\
      & {\begin{align}\label{eq7}
      \implies \myvec{r\\s} = \myvec{\frac{-1}{3}\\\frac{1}{3}}
      \end{align}}\\
      &\\
      & Substitute \eqref{eq5} and \eqref{eq7} in \eqref{eq3} we get;\\
      &{\begin{align}\label{eq8}
      & \myvec{x\\y}: \myvec{\frac{2}{3}\\\frac{1}{3}}x
      +\myvec{\frac{-1}{3}\\\frac{1}{3}}y = \myvec{\frac{2}{3}&\frac{-1}{3}\\\frac{1}{3}&\frac{1}{3}}\myvec{x\\y}
      \end{align}}\\
      &\\
      & From \eqref{eq1} and \eqref{eq8} ; \\
      & {\begin{align} \label{eq9} 
      \myvec{x\\y} : \myvec{\frac{2}{3} &\frac{-1}{3}}\myvec{1\\-1}\myvec{x\\y}+\myvec{\frac{1}{3} &\frac{1}{3}}\myvec{1\\2}\myvec{x\\y}
      \end{align}}\\
      &\\
      & From \eqref{eq2} and \eqref{eq9};\\ 
      & {\begin{align} \label{eq10} 
      \vec{E}\myvec{x\\y} = \myvec{\frac{2}{3}x-\frac{1}{3}y} \myvec{1\\-1}
      \end{align}}\\
      & \\
      &{\begin{align}
     \implies \vec{E}\myvec{x\\y} = \myvec{\frac{2}{3}\\\frac{-2}{3}}x + \myvec{\frac{-1}{3}\\\frac{1}{3}}y
     \end{align}}\\
     &{\begin{align}
     \vec{E}\myvec{x\\y} = \myvec{\frac{2}{3}&\frac{-1}{3}\\\frac{-2}{3}&\frac{1}{3}}\myvec{x\\y}
      \end{align}}\\
     & Hence,\\
      &{\begin{align}
      \implies{ \vec{E}= \myvec{\frac{2}{3}&\frac{-1}{3}\\\frac{-2}{3}& \frac{1}{3}}}
      \end{align}}\\
      &\\
\hline
\multirow{3}{*}{Verification} & \\
& If $n\times n$ matrix $\vec{E}$ is projecion matrix, then\\
	&\\
	&$\qquad\qquad\qquad \vec{E}^2=\vec{E}$\\
	&\\
  &{\begin{align}
  \vec{E^{2}} = \myvec{\frac{2}{3}&\frac{-1}{3}\\\frac{-2}{3}& \frac{1}{3}}\myvec{\frac{2}{3}&\frac{-1}{3}\\\frac{-2}{3}& \frac{1}{3}}
  \implies \vec{E}=\vec{E^{2}}=\myvec{\frac{2}{3}&\frac{-1}{3}\\\frac{-2}{3}& \frac{1}{3}}
  \end{align}}\\
  & Hence, Verified.\\
  &\\
	\hline
\caption{Finding Projection Matrix}
\label{table:1}
\end{longtable}
\end{document}
