\documentclass[journal,12pt]{IEEEtran}
\usepackage{longtable}
\usepackage{setspace}
\usepackage{gensymb}
\singlespacing
\usepackage[cmex10]{amsmath}
\newcommand\myemptypage{
	\null
	\thispagestyle{empty}
	\addtocounter{page}{-1}
	\newpage
}
\usepackage{amsthm}
\usepackage{mdframed}
\usepackage{mathrsfs}
\usepackage{txfonts}
\usepackage{stfloats}
\usepackage{bm}
\usepackage{cite}
\usepackage{cases}
\usepackage{subfig}

\usepackage{longtable}
\usepackage{multirow}

\usepackage{enumitem}
\usepackage{mathtools}
\usepackage{steinmetz}
\usepackage{tikz}
\usepackage{circuitikz}
\usepackage{verbatim}
\usepackage{tfrupee}
\usepackage[breaklinks=true]{hyperref}
\usepackage{graphicx}
\usepackage{tkz-euclide}

\usetikzlibrary{calc,math}
\usepackage{listings}
    \usepackage{color}                                            %%
    \usepackage{array}                                            %%
    \usepackage{longtable}                                        %%
    \usepackage{calc}                                             %%
    \usepackage{multirow}                                         %%
    \usepackage{hhline}                                           %%
    \usepackage{ifthen}                                           %%
    \usepackage{lscape}     
\usepackage{multicol}
\usepackage{chngcntr}

\DeclareMathOperator*{\Res}{Res}

\renewcommand\thesection{\arabic{section}}
\renewcommand\thesubsection{\thesection.\arabic{subsection}}
\renewcommand\thesubsubsection{\thesubsection.\arabic{subsubsection}}

\renewcommand\thesectiondis{\arabic{section}}
\renewcommand\thesubsectiondis{\thesectiondis.\arabic{subsection}}
\renewcommand\thesubsubsectiondis{\thesubsectiondis.\arabic{subsubsection}}


\hyphenation{op-tical net-works semi-conduc-tor}
\def\inputGnumericTable{}                                 %%

\lstset{
%language=C,
frame=single, 
breaklines=true,
columns=fullflexible
}
\begin{document}
\onecolumn

\newtheorem{theorem}{Theorem}[section]
\newtheorem{problem}{Problem}
\newtheorem{proposition}{Proposition}[section]
\newtheorem{lemma}{Lemma}[section]
\newtheorem{corollary}[theorem]{Corollary}
\newtheorem{example}{Example}[section]
\newtheorem{definition}[problem]{Definition}

\newcommand{\BEQA}{\begin{eqnarray}}
\newcommand{\EEQA}{\end{eqnarray}}
\newcommand{\define}{\stackrel{\triangle}{=}}
\bibliographystyle{IEEEtran}
\raggedbottom
\setlength{\parindent}{0pt}
\providecommand{\mbf}{\mathbf}
\providecommand{\pr}[1]{\ensuremath{\Pr\left(#1\right)}}
\providecommand{\qfunc}[1]{\ensuremath{Q\left(#1\right)}}
\providecommand{\sbrak}[1]{\ensuremath{{}\left[#1\right]}}
\providecommand{\lsbrak}[1]{\ensuremath{{}\left[#1\right.}}
\providecommand{\rsbrak}[1]{\ensuremath{{}\left.#1\right]}}
\providecommand{\brak}[1]{\ensuremath{\left(#1\right)}}
\providecommand{\lbrak}[1]{\ensuremath{\left(#1\right.}}
\providecommand{\rbrak}[1]{\ensuremath{\left.#1\right)}}
\providecommand{\cbrak}[1]{\ensuremath{\left\{#1\right\}}}
\providecommand{\lcbrak}[1]{\ensuremath{\left\{#1\right.}}
\providecommand{\rcbrak}[1]{\ensuremath{\left.#1\right\}}}
\theoremstyle{remark}
\newtheorem{rem}{Remark}
\newcommand{\sgn}{\mathop{\mathrm{sgn}}}
\providecommand{\abs}[1]{\left\vert#1\right\vert}
\providecommand{\res}[1]{\Res\displaylimits_{#1}} 
\providecommand{\norm}[1]{\left\lVert#1\right\rVert}
%\providecommand{\norm}[1]{\lVert#1\rVert}
\providecommand{\mtx}[1]{\mathbf{#1}}
\providecommand{\mean}[1]{E\left[ #1 \right]}
\providecommand{\fourier}{\overset{\mathcal{F}}{ \rightleftharpoons}}
%\providecommand{\hilbert}{\overset{\mathcal{H}}{ \rightleftharpoons}}
\providecommand{\system}{\overset{\mathcal{H}}{ \longleftrightarrow}}
	%\newcommand{\solution}[2]{\textbf{Solution:}{#1}}
\newcommand{\solution}{\noindent \textbf{Solution: }}
\newcommand{\cosec}{\,\text{cosec}\,}
\providecommand{\dec}[2]{\ensuremath{\overset{#1}{\underset{#2}{\gtrless}}}}
\newcommand{\myvec}[1]{\ensuremath{\begin{pmatrix}#1\end{pmatrix}}}
\newcommand{\mydet}[1]{\ensuremath{\begin{vmatrix}#1\end{vmatrix}}}
\numberwithin{equation}{subsection}
\makeatletter
\@addtoreset{figure}{problem}
\makeatother
\let\StandardTheFigure\thefigure
\let\vec\mathbf
\renewcommand{\thefigure}{\theproblem}
\def\putbox#1#2#3{\makebox[0in][l]{\makebox[#1][l]{}\raisebox{\baselineskip}[0in][0in]{\raisebox{#2}[0in][0in]{#3}}}}
     \def\rightbox#1{\makebox[0in][r]{#1}}
     \def\centbox#1{\makebox[0in]{#1}}
     \def\topbox#1{\raisebox{-\baselineskip}[0in][0in]{#1}}
     \def\midbox#1{\raisebox{-0.5\baselineskip}[0in][0in]{#1}}
\vspace{3cm}
\title{Assignment 17}
\author{Neha Rani\\EE20MTECH14014}
\maketitle
\bigskip
\renewcommand{\thefigure}{\theenumi}
\renewcommand{\thetable}{\theenumi}
%
Download the latex-tikz codes from 
%
\begin{lstlisting}
https://github.com/neharani289/MatrixTheory/Assignment17
\end{lstlisting}
\section{\textbf{Problem}}
%
 (ugcdec/2015/72) :\\
Let $\vec{V}$ be the vector space of polynomials over $\mathbb{R}$ of degree less than or equal to $n$. For $p(x)=a_0+a_{n-1}x+..+a_nx^{n}$ in $\vec{V}$, define a linear transformation $\vec{T}:\vec{V}\rightarrow \vec{V}$ by $\brak{\vec{T}p}(x)=a_n+a_{n-1}x+..+a_0x^n.$ Then \\
\begin{enumerate}
    \item $\vec{T}$ is one to one.
    \item $\vec{T}$ is onto.
    \item $\vec{T}$ is invertible.
    \item $\det{\vec{T}}=\pm 1$.
\end{enumerate}
\section{\textbf{Definition and theorem used}}
\renewcommand{\thetable}{1}
\begin{table}[ht!]
\centering
\begin{tabular}{|p{3cm}|p{15cm}|}
    \hline
	\multirow{3}{*}{Theorem} 
	&\\
	& Suppose $T:\mathbb{R}^{n}\rightarrow \mathbb{R}^{m}$ is the linear transformation $\vec{T}(\vec{x})=\vec{A}\vec{x}$ where $\vec{A}$ s an $m\times n$ matrix.\\
	&\\
	&{\begin{enumerate}
	    \item $T$ is \textbf{one to one} if the columns of $\vec{A}$ are linearly independent , which happens precisely when $\vec{A}$ has a pivot position in every column.
	    
	    \item $T$ is \textbf{onto} if an over $\mathbb{R}$  only if the span of the columns of $\vec{A}$ is $\mathbb{R}^{n}$, which happens precisely when $\vec{A}$ has a pivot position in every row.
	\end{enumerate}}\\
	\hline
	\multirow{3}{*}{$Range(\vec{T})$}
	&\\
	& It is column-space of linear operator $\vec{T}$.\\
	&\\
    &$\qquad\qquad\qquad\vec{T}(\vec{x})=\vec{v}
    \implies\vec{Ax}=\vec{v}$\\
    &\\
    & where $\vec{x}$,$\vec{v}\in\vec{V}$ and We can also say that\\
    &\\
    &$\qquad\qquad\qquad Range(\vec{T})=C(\vec{A})\label{R}$\\
    &where $C(\vec{A})$ is column space of $\vec{A}$.\\
    \hline
    \multirow{3}{*}{$rank(\vec{T})$}
    &\\
&$rank(\vec{T})=rank(\vec{A})$\\
&\\
    \hline
\end{tabular}
\label{table:1}
    \caption{Definitions and Theorem }
\end{table}
\newpage
\section{\textbf{Solution}}
\renewcommand{\thetable}{1}
\begin{longtable}{|p{5cm}|p{13cm}|}
\hline
\endhead
    \multirow{3}{*}{\textbf{Given}} 
     &\\
     & $\vec{V}$ be a vector space of polynomials over $\mathbb{R}$ of degree less then $n$ \\
     &\\
     &$\qquad\qquad\qquad p(x)=a_0+a_{n-1}x+..+a_nx^{n}$\\
     &\\
     & $\vec{T}: \vec{V} \rightarrow \vec{V}$\\
     &\\
     & $\qquad\qquad\qquad(\vec{T}p)(x)=a_n+a_{n-1}x+..+a_0x^{n}$\\
     &\\
     \hline
     \multirow{3}{*}{\textbf{Explanation}}&\\
     & We know that Basis for a polynomial vector space $P=\brak{p_1,p_2,..,p_n}$ is a set of vectors that spans the space, and is linearly independent .\\
     &\\
     &$\qquad\qquad$  Basis = $\brak{1,x,x^2,...,x^{n}}$\\
     &\\
     & $\qquad\qquad\vec{T}(1) = x^{n} = 0.1+0.x+..+0.x^{n-1}+1.x^n$\\
     & $\qquad\qquad\vec{T}(x) = x^{n-1} = 0.1+0.x+..+1.x^{n-1}+0.x^{n}$\\
     & $\qquad\qquad\vdots$\\
     & $\qquad\qquad\vec{T}(x^{n}) = 1 = 1.1+0.x+..+0.x^{n-1}+0.x $\\
     &\\
     & Expressing $\vec{T}$ in matrix form\\
     &\\
     & $\qquad\qquad\qquad\vec{T}=\myvec{0&0&\cdots&0&1\\0&0&\cdots&1&0\\\vdots&\vdots&\ddots&\vdots&\vdots\\0&1&\cdots&0&0\\1&0&\cdots&0&0}$\\ 
     &\\
    \hline
    \multirow{3}{*}{\textbf{Example} }
	& \\
	& For Simplicity , Let $n=3$\\
	&\\
	& $\qquad\qquad\qquad\implies p(x)=a_0+a_1x+a_2x^{2}+a_3x^{3}$\\
	&\\
	& $\qquad\qquad\qquad\implies (\vec{T})p(x)=a_3+a_2x+a_1x^2+a_0x^3$\\
	&\\
	& Basis = $\brak{1,x,x^2,x^3}$\\
	&\\
    & $\qquad\qquad\vec{T}(1) = 0.0+0.x+0.x^2+1.x^3$\\
    &\\
    & $\qquad\qquad\vec{T}(x) = 0.0+0.x+1.x^2+0.x^3$\\
    &\\
    & $\qquad\qquad\vec{T}(x^2) = 0.0+1.x+0.x^2+0.x^3$\\
    &\\
    & $\qquad\qquad\vec{T}(x^3) = 1.1+0.x+0.x^2+0.x^3$\\
    &\\
    & Expressing $\vec{T}$ in matrix form;\\
\hline
    & $\qquad\qquad\qquad\vec{T}=\myvec{0&0&0&1\\0&0&1&0\\0&1&0&0\\1&0&0&0}$\\
	&\\
	\hline
	\multirow{3}{*}{\textbf{Statement 1}:$\vec{T}$ is one to one } & \\
	& True\\
	\hline
	&\\
	& $\vec{T}:\vec{V}\rightarrow\vec{V}$ be a linear transformation\\
	&\\
	& $\vec{T}$ is one-to-one if and only if the nullity of $\vec{T}$ is zero.\\
	&\\
	& According to rank-nullity theorem.\\
    & $\qquad\qquad dim(\vec{V})=rank(\vec{T})+nullity(\vec{T})$\\
    &\\
    & $\qquad\qquad\qquad\vec{T}=\myvec{0&0&0&1\\0&0&1&0\\0&1&0&0\\1&0&0&0}$\\
	&\\
	& Here, $\qquad\qquad dim(\vec{V}) = 4$\\
	&\\
	& $\qquad\qquad rank(\vec{T})=$ no. of linearly independent column or row $=4$\\.
	&\\
	& $\qquad\qquad\implies nullity (\vec{T}) = 0$\\
	&\\
	& Thus, we can conclude $\vec{T}$ is one to one .\\
	  
	\hline
	\multirow{3}{*}{\textbf{Statement 2}:$\vec{T}$ is onto}&\\
    & True\\
	\hline
	&\\
	& A matrix transformation is onto if and only if the matrix has a pivot position in each row, if the number of pivots is equal to the number of rows.\\
	&\\
	& $\qquad\qquad\qquad\vec{T}=\myvec{0&0&0&1\\0&0&1&0\\0&1&0&0\\1&0&0&0}$\\
	&\\
	& $\qquad\qquad\implies rank(\vec{T})= 4$ which is equal to no of rows.\\
	&\\
	& Thus, we can conclude $\vec{T}$ is onto.\\
	&\\
	
	\hline
	\multirow{3}{*}{\textbf{Statement 3}:$\vec{T}$ is invertible} &\\
    & True\\
	\hline
	&\\
	& $T:\mathbb{R}^{n}\rightarrow \mathbb{R}^{m}$ is the linear transformation $\vec{T}(\vec{x})=\vec{A}\vec{x}$ where $\vec{A}$ s an $m\times n$ matrix , then $\vec{T}$ is invertible if and only if $\vec{A}$ is invertible.\\
	&\\
	& $\qquad\qquad\qquad\vec{T}=\myvec{0&0&0&1\\0&0&1&0\\0&1&0&0\\1&0&0&0} , \det{\vec{T}}\neq 0$\\
	&\\
	& Thus, we can conclude $\vec{T}$ is invertible.\\
	&\\
	\hline
	\multirow{3}{*}{Statement 4: $\det{\vec{T}}=\pm1$}&\\
	& True\\
	\hline
	&\\
		& $\qquad\qquad\qquad\vec{T}=\myvec{0&0&0&1\\0&0&1&0\\0&1&0&0\\1&0&0&0}$ , where $\vec{T}$ is a permutation matrix .\\
	&\\
	
	& A permutation matrix is nonsingular matrix, and determinant is $\pm 1$. Permutation matrix $\vec{A}$ satisfies $\vec{A}{\vec{A}^{T}}=\vec{I}$\\
	
	&\\
	& Here, $\qquad\qquad\vec{T}\vec{T}^{T}=\myvec{0&0&0&1\\0&0&1&0\\0&1&0&0\\1&0&0&0}\myvec{0&0&0&1\\0&0&1&0\\0&1&0&0\\1&0&0&0}$\\
	&\\
	&$\qquad\qquad\qquad\vec{T}\vec{T}^{T}=\myvec{1&0&0&0\\0&1&0&0\\0&0&1&0\\0&0&0&1} = \vec{I}$ , also an Involutory matrix .\\
	&\\
	& Thus, we can conclude $\det{\vec{T}}=\pm1$\\
	&\\
	\hline
	\caption{Solution Summary}
    \label{table:1}
\end{longtable}
\end{document}

