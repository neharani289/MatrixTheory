\documentclass[journal,12pt]{IEEEtran}
\usepackage{longtable}
\usepackage{setspace}
\usepackage{gensymb}
\singlespacing
\usepackage[cmex10]{amsmath}
\newcommand\myemptypage{
	\null
	\thispagestyle{empty}
	\addtocounter{page}{-1}
	\newpage
}
\usepackage{amsthm}
\usepackage{mdframed}
\usepackage{mathrsfs}
\usepackage{txfonts}
\usepackage{stfloats}
\usepackage{bm}
\usepackage{cite}
\usepackage{cases}
\usepackage{subfig}

\usepackage{longtable}
\usepackage{multirow}

\usepackage{enumitem}
\usepackage{mathtools}
\usepackage{steinmetz}
\usepackage{tikz}
\usepackage{circuitikz}
\usepackage{verbatim}
\usepackage{tfrupee}
\usepackage[breaklinks=true]{hyperref}
\usepackage{graphicx}
\usepackage{tkz-euclide}

\usetikzlibrary{calc,math}
\usepackage{listings}
    \usepackage{color}                                            %%
    \usepackage{array}                                            %%
    \usepackage{longtable}                                        %%
    \usepackage{calc}                                             %%
    \usepackage{multirow}                                         %%
    \usepackage{hhline}                                           %%
    \usepackage{ifthen}                                           %%
    \usepackage{lscape}     
\usepackage{multicol}
\usepackage{chngcntr}

\DeclareMathOperator*{\Res}{Res}

\renewcommand\thesection{\arabic{section}}
\renewcommand\thesubsection{\thesection.\arabic{subsection}}
\renewcommand\thesubsubsection{\thesubsection.\arabic{subsubsection}}

\renewcommand\thesectiondis{\arabic{section}}
\renewcommand\thesubsectiondis{\thesectiondis.\arabic{subsection}}
\renewcommand\thesubsubsectiondis{\thesubsectiondis.\arabic{subsubsection}}


\hyphenation{op-tical net-works semi-conduc-tor}
\def\inputGnumericTable{}                                 %%

\lstset{
%language=C,
frame=single, 
breaklines=true,
columns=fullflexible
}
\begin{document}
\onecolumn

\newtheorem{theorem}{Theorem}[section]
\newtheorem{problem}{Problem}
\newtheorem{proposition}{Proposition}[section]
\newtheorem{lemma}{Lemma}[section]
\newtheorem{corollary}[theorem]{Corollary}
\newtheorem{example}{Example}[section]
\newtheorem{definition}[problem]{Definition}

\newcommand{\BEQA}{\begin{eqnarray}}
\newcommand{\EEQA}{\end{eqnarray}}
\newcommand{\define}{\stackrel{\triangle}{=}}
\bibliographystyle{IEEEtran}
\raggedbottom
\setlength{\parindent}{0pt}
\providecommand{\mbf}{\mathbf}
\providecommand{\pr}[1]{\ensuremath{\Pr\left(#1\right)}}
\providecommand{\qfunc}[1]{\ensuremath{Q\left(#1\right)}}
\providecommand{\sbrak}[1]{\ensuremath{{}\left[#1\right]}}
\providecommand{\lsbrak}[1]{\ensuremath{{}\left[#1\right.}}
\providecommand{\rsbrak}[1]{\ensuremath{{}\left.#1\right]}}
\providecommand{\brak}[1]{\ensuremath{\left(#1\right)}}
\providecommand{\lbrak}[1]{\ensuremath{\left(#1\right.}}
\providecommand{\rbrak}[1]{\ensuremath{\left.#1\right)}}
\providecommand{\cbrak}[1]{\ensuremath{\left\{#1\right\}}}
\providecommand{\lcbrak}[1]{\ensuremath{\left\{#1\right.}}
\providecommand{\rcbrak}[1]{\ensuremath{\left.#1\right\}}}
\theoremstyle{remark}
\newtheorem{rem}{Remark}
\newcommand{\sgn}{\mathop{\mathrm{sgn}}}
\providecommand{\abs}[1]{\left\vert#1\right\vert}
\providecommand{\res}[1]{\Res\displaylimits_{#1}} 
\providecommand{\norm}[1]{\left\lVert#1\right\rVert}
%\providecommand{\norm}[1]{\lVert#1\rVert}
\providecommand{\mtx}[1]{\mathbf{#1}}
\providecommand{\mean}[1]{E\left[ #1 \right]}
\providecommand{\fourier}{\overset{\mathcal{F}}{ \rightleftharpoons}}
%\providecommand{\hilbert}{\overset{\mathcal{H}}{ \rightleftharpoons}}
\providecommand{\system}{\overset{\mathcal{H}}{ \longleftrightarrow}}
	%\newcommand{\solution}[2]{\textbf{Solution:}{#1}}
\newcommand{\solution}{\noindent \textbf{Solution: }}
\newcommand{\cosec}{\,\text{cosec}\,}
\providecommand{\dec}[2]{\ensuremath{\overset{#1}{\underset{#2}{\gtrless}}}}
\newcommand{\myvec}[1]{\ensuremath{\begin{pmatrix}#1\end{pmatrix}}}
\newcommand{\mydet}[1]{\ensuremath{\begin{vmatrix}#1\end{vmatrix}}}
\numberwithin{equation}{subsection}
\makeatletter
\@addtoreset{figure}{problem}
\makeatother
\let\StandardTheFigure\thefigure
\let\vec\mathbf
\renewcommand{\thefigure}{\theproblem}
\def\putbox#1#2#3{\makebox[0in][l]{\makebox[#1][l]{}\raisebox{\baselineskip}[0in][0in]{\raisebox{#2}[0in][0in]{#3}}}}
     \def\rightbox#1{\makebox[0in][r]{#1}}
     \def\centbox#1{\makebox[0in]{#1}}
     \def\topbox#1{\raisebox{-\baselineskip}[0in][0in]{#1}}
     \def\midbox#1{\raisebox{-0.5\baselineskip}[0in][0in]{#1}}
\vspace{3cm}
\title{Assignment 13}
\author{Neha Rani\\EE20MTECH14014}
\maketitle
\bigskip
\renewcommand{\thefigure}{\theenumi}
\renewcommand{\thetable}{\theenumi}
%
Download the latex-tikz codes from 
%
\begin{lstlisting}
https://github.com/neharani289/MatrixTheory/Assignment13
\end{lstlisting}
\bigskip
\section{\textbf{Problem}}
(hoffman/page198/9) : \\
%
Let $\vec{A}$ be an $n\times n$ matrix with characteristics polynomial\\
\begin{align}
& f = (x-c_1)^{d_1}....(x-c_k)^{d_k} \nonumber
\end{align}
Show that
\begin{align}
& c_1d_1+....+c_kd_k = trace(A)\nonumber
\end{align}
%
%
%
\bigskip\\
\section{\textbf{Solution}}
\renewcommand{\thetable}{1}
\begin{longtable}{|l|l|}
\hline
\multirow{3}{*}{Given} & \\
& Let $\vec{A}$ be an $n\times n$ \\
&\\
& $\vec{A} = \myvec{a_{11}&a_{12}&\cdots &a_{1n} \\
a_{21}&a_{22}&\cdots &a_{2n} \\
\vdots & \vdots & \ddots & \vdots\\
a_{n1}&a_{n2}&\cdots &a_{nn}}$\\
&\\
& and Characteristics polynomial\\
&\\
&  $f= (x-c_1)^{d_1}....(x-c_k)^{d_k}$\\
&\\
\hline
\multirow{3}{*}{To prove} & \\
& $c_1d_1+....+c_kd_k = trace(A)$\\
&\\
\hline
\multirow{3}{*}{proof} & \\
& Characteristics polynomial; \\
&  $f= (x-c_1)^{d_1}....(x-c_k)^{d_k}$\\
& here,\\
& $c_1,...,c_k$ are the distinct eigen values.\\
& and $d_1,...,d_k$ denotes the repetition of eigen values\\
&\\
& Therefore,\\
& $\boxed{c_1d_1+c_2d_2+...+c_kd_k = \sum_{i} \lambda_i}$ =Sum of all eigen values.\\
&\\


& Using JCF concept;\\
&\\
& For every matrix $\vec{A}$ there exist a invertible matrix $\vec{P}$ such that\\
& $\vec{PAP^{-1} = J}$ where $J$ has Jordan Canonical form.\\ 
&\\
& $\vec{J}=\myvec{J_1& & & \\& J_2 & \\& &  \ddots \\& & & J_p}$ ,each block $J_i$ is a square matrix of the form \\
&\\
& $\vec{J_i}= \myvec{\lambda_i&1 & &\\ &\lambda_i&\ddots&\\&&\ddots &1\\&&&\lambda_i}$ \\
&where, J is called Jordan normal form of $\vec{A}$ and $\vec{J_i}$ is called a Jordan block of $\vec{A}$\\
&\\
& Consider, two $n\times n$ matrices B and C and they are  called similar if\\
& there exists an invertible $n\times n$ matrix P such that $\vec{C=PBP^{-1}}$\\ 
& which implies $B$ and $C$ have the same eigen values. \\
&\\
& $\implies trace(ABC)=trace(BAC)=trace(CAB)$\\ 
&\\ 
& $\implies trace(A)=trace(P^{-1}JP)=trace(PP^{-1}J)=trace(J)$\\
&\\
& $\boxed{trace(J)=\sum_{i} \lambda_i}$ where, $\lambda_i$ are eigen values of $\vec{A}$.\\
&\\
& $\boxed{\implies c_1d_1+c_2d_2+...+c_kd_k = \sum_{i} \lambda_i = trace{A}}$\\
&\\
& Hence, Proved.\\
&\\
\hline
\caption{Solution Summary}
\label{table:1}
\end{longtable}
\end{document}